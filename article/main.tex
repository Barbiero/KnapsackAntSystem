\documentclass[12pt,a4paper]{article}

\usepackage[utf8]{inputenc}
\usepackage[brazilian,english]{babel}
\usepackage{amsmath}
\usepackage{amssymb}
\usepackage[hidelinks]{hyperref}
\usepackage[ruled,vlined,algosection]{algorithm2e}
\AtBeginDocument{\let\textlabel\label}


\begin{document}

\selectlanguage{brazilian}
\title{Paralelização do Algoritmo de Colonização de Formigas com o Problema da Mochila de Múltiplas Restrições}

\author{Pedro Barbiero\\
	Universidade Estadual de Maringá\\
	{\tt pedrobarbiero@gmail.com}
}

\maketitle

\selectlanguage{english}
\begin{abstract}
	This report describes the process of parallelizing the Ant Colonization Optimization algorithm, used to solve the Multiple-Constraint Knapsack problem, through a distributed memory model. It was noted that, given a sufficiently small message size that the overhead from this model is negligible, achieving very high speedup factors. It was also noted that at certain instances, there was a super-linear speedup and it was hipotesized that this happens due to the hyper-threading technology provided on the Intel i7 processor, but no conclusion was reached from this phenomenom.
	
\end{abstract}

\selectlanguage{brazilian}

\section{Introdução}
	O algoritmo de otimização da colônia de formigas(sigla em inglês, {\it ACO}) é uma heurística baseada em probabilidade no qual se baseia no conceito de formigas que buscam alimentos e deixam trilhas de feromônio para que outras formigas encontrem este mesmo alimento\cite{antsystem}. Em termos de computacionais, considera-se que as \textit{formigas} são elementos que irão atravessar um grafo de um problema, enquanto a \textit{comida} será uma solução válida para este problema, tal que as formigas irão encontrar uma solução boa(e possívelmente, ótima) para o dado problema.
	
	O problema em questão é nomeado \textit{Problema da Mochila de Múltiplas Restrições}, em inglês \textit{Multiple Knapsack Problem}. Este problema é uma variação do problema da mochila onde cada nó do grafo do problema representa um estado da mochila, e cada aresta representa a introdução de um novo item na mochila. Este problema foi mostrado como NP-Completo por Gens, G\cite{complexity}.
	
	Neste trabalho, iremos utilizar o \textit{ACO} para resolver o problema em questão, e então melhorar o tempo de solução do problema criando múltiplas instâncias do programa de solução, tais que elas se comuniquem e resolvam o problema da mesma forma que em um programa sequencial, podendo ser escalada para diversas máquinas ou mesmo em diversos processadores de uma única máquina.

\section{Problema da Mochila de Múltiplas Restrições}

\subsection{O Problema da Mochila}
O Problema da Mochila tradicional é um problema de otimização combinatória, normalmente descrito como  \textbf{"Dado um conjunto de possíveis itens, cada um com um certo valor e um certo peso, e uma mochila com uma capacidade de peso limite, qual é a melhor combinação de itens que a mochila consegue suportar, tal que o valor seja maximizado?"}.

Matematicamente, o problema é descrito como:
	\begin{displaymath}
	\max \sum_{i=1}^{n} v_i x_i
	\end{displaymath}
	\begin{displaymath}
	tal\ que \sum_{i=1}^{n} w_i x_i \leq W
	\end{displaymath}
	Onde 
	\begin{itemize}
		\item $n$ é a quantidade de itens que podem ser postos na mochila,
		\item $v_i$ é o valor do item $i$,
		\item  $w_i$ é o peso do item $i$, 
		\item $W$ é a capacidade da mochila e
		\item $x_i$ é a quantidade de vezes em que o item $i$ é colocado na mochila.
	\end{itemize}
	
Em particular, a variação mais comum do problema a ser encontrado é a \textit{Mochila Binária}, onde $x_i \in \{0,1\}$, ou seja, existe apenas um item de cada tipo que pode ser posto na mochila, e este item não pode ser fracionado.

\subsection{Mochila Binária com Múltiplas Restrições}
Outra variação do problema da mochila binária determina que a mochila possui mais do que uma restrição - por exemplo, além da capacidade de peso existe um limite de volume. O problema é então descrito na forma
\begin{eqnarray}
\max & \sum_{i=1}^{m} \sum_{j=1}^{n} p_{ij} x_{ij} & \nonumber\\
tal\ que & \sum_{j=1}^{n} w_j x_{ij} \leq W_i, & \forall\ 1 \leq i \leq m,\nonumber\\
		& \sum_{i=1}^{m} x_{ij} \leq 1, & \forall\ 1 \leq j \leq n, \nonumber\\
		& x_{ij} \in \{0,1\}, & (\forall\ 1 \leq j \leq n )\wedge (\forall\ 1 \leq i \leq m) \nonumber
\end{eqnarray}




\section{Sistema de Colonização de Formigas}
O Algoritmo de Colonização de Formigas, ou \textit{ACO}, é uma meta-heurística de otimização aplicada em diversas áreas e é útil quando um tempo curto é necessário para resolver um problema \cite{acomkp}.  É um sistema onde múltiplos agentes(formigas) agem de uma forma relativamente simples porém interagem entre si(através de um feromônio) resultando em um comportamento complexo.

O ACO consiste em repetir diversas vezes um conjunto de operações:
\begin{enumerate}
	\item Buscar solução baseado no feromônio atual
	\item Atualizar o Feromônio
	\item (opcional) Operações de \textit{Daemon}
\end{enumerate}

Estas operações são repetidas várias vezes até que uma condição de parada seja satisfeita: um limite no número de iterações, tempo ou mesmo uma interrupção externa. Em geral, espera-se que o algoritmo encontre uma melhor solução se este ciclo for repetido mais vezes.

\subsection{Construção da Solução}
Como é comum para o ACO tratar o problema como um grafo, a construção da solução pode ser feita pelo Algoritmo \ref{buildSol}.

\begin{algorithm}[hb]
	$Solucao \gets \emptyset$\;

	\While{true}{
		$A \gets $ conjunto de arestas que levam a uma solução válida\;
		\If{$A = \emptyset $}{
			$return$ \textbf{Solucao}\;
		}
			
		Solucao avança em uma aresta de A\;
	}
	\caption{Algoritmo de construcão de solução}
	\label{buildSol}
\end{algorithm}

Embora o algoritmo pareça relativamente simples, o método de seleção da aresta que será avançada é o diferencial do \textit{ACO}. Neste sistema, atribuem-se a cada aresta dois valores chamados \textit{feromônio}($\tau$) e \textit{atratividade}($\mu$). Estes valores podem, ou não, ser relativos ao estado atual da solução no momento de sua construção, e relativos aos atributos das arestas; A seleção é feita então probabilisticamente: Dado um conjunto $A_i$ de possíveis arestas para construir a solução em um passo $i$, a chance de selecionar uma aresta $j$ é dada por
\[ 
	p_j = 
	\begin{cases}
	\dfrac{\tau_j^\alpha \mu_j^\beta}
	{\sum_{k \in A_i} \tau_k^\alpha \mu_k^\beta},
	 & \text{se } j \in A_i,\\
	 0 & \text{se } j \notin A_i.
	\end{cases}
\]
O valor de $\mu_j$ é dado por uma combinação de fatores do problema; no problema da mochila, por exemplo, pode se referir a uma relação entre os atributos de um item e a capacidade total da mochila. Este valor deve ser determinado como uma heurística para ajudar a selecionar as melhores arestas para a solução. As constantes $\alpha$ e $\beta$ determinam o peso que o feromônio $\tau$ e a atratividade $\mu$ terão, respectivamente.

\subsection{Atualização do Feromônio}
Em cada iteração do sistema, o feromônio será atualizado de acordo com a melhor solução encontrada por uma formiga nesta iteração, assim aumentando a probabilidade de seguir um caminho de uma boa solução para a próxima iteração. Além de existir a soma do feromônio, ele também irá \textit{evaporar}, isto é, o valor numérico do feromônio será multiplicado por uma constante $0<\rho<1$. Assim, a atualização do feromônio é dada pelo algoritmo \ref{pherUpdate}.

\begin{algorithm}[ht]
	\KwIn{$S_k$: Conjunto de soluções encontradas por todas as formigas\\
		$\rho$: Taxa de evaporação do feromônio, $0\leq\rho\leq1$}
	
	\ForEach{$j :$ aresta no sistema}{
		$\tau_j \gets \tau_j * \rho$\;
	}
	
	\ForEach{$S\in S_k$}{
		\ForEach{$i$: Aresta $\in S$}{
			$\tau_i \gets \tau_i * \Delta\tau(S)$\;
		}	
	}
	\caption{Atualização de Ferômonio}
	\label{pherUpdate}
\end{algorithm}

Note que $\Delta\tau$ é um valor que pode, ou não, depender de atributos da solução S, mas é comum que este valor seja relativo à qualidade da solução encontrada - assim, soluções melhores colocam mais feromônio no sistema.

\subsection{Operações de Daemon}
São operações dependentes da implementação. Por exemplo, o cálculo de $\tau_j^\alpha \mu_j^\beta$ é computacionalmente custoso, e pode ser útil atualizá-lo em cada aresta apenas uma vez por iteração ao invés de calculá-lo em todas as chamadas, assim otimizando a velocidade de execução do programa.




\section{Paralelização do Algoritmo em Memória Distribuída}
Programas sequenciais podem ser programados, com a ajuda de uma biblioteca específica, de tal forma que eles possam ser executados em múltiplas instâncias através de diversas máquinas interligadas em rede(local ou internet). A \textbf{Message Passing Interface}, ou MPI, é o modelo de bibliotecas para um sistema de passagem de mensagens; Neste trabalho em específico foi-se utilizada a biblioteca \texttt{mpich}, que implementa este modelo.

\subsection{Conceitos básicos do MPI}
Alguns conceitos precisam ser compreendidos para entender como o MPI funciona, e como o programa pode tirar vantagem destes conceitos para resolver problemas sobre múltiplas máquinas ou processadores. Primeiramente, deve-se entender como \textbf{nó} uma instância de um programa que pode se comunicar com outros nós de mesmo tipo; Esta comunicação não é detalhada pois pode ocorrer internamente(na mesma máquina) ou externamente(entre diferentes máquinas), e ela pode ocorrer de acordo com o protocolo de comunicação definido pelo MPI - para o programa e o algoritmo em questão, esta comunicação é abstraída e tratada pela biblioteca.

\subsubsection{Comunicadores}
No MPI, um \textbf{comunicador} é um conjunto de um ou mais nós que fazem parte, de forma lógica, de um mesmo grupo de comunicação. Comunicadores são utilizados para determinar a topologia dos nós em execução assim como organizar as passagens de mensagem(principalmente as do tipo \textit{broadcast}). Por padrão, um comunicador global chamado \texttt{MPI\_COMM\_WORLD} é definido na inicialização dos nós, e este comunicador irá englobar todos os nós do sistema.

Neste projeto, foi-se utilizada uma topologia do tipo \textbf{estrela}; Isto significa que todos os nós irão passar mensagens entre todos os outros nós, significando que o comunicador \texttt{MPI\_COMM\_WORLD} será o único criado.

\subsubsection{Rank \& Size}
A identificação de um dado nó é dada por seu \textbf{rank} dentro de um comunicador; O comunicador, por sua vez, possui um número de nós variável que é dado por \textbf{size}, e cada nó pertencente a um comunicador obedece à lei de $0\leq rank_{no}<Size_{comm}$.

Neste projeto, um rank em particular, o rank $0$, é especificado como o nó que irá produzir a saída de dados; Além de imprimir na tela(ou arquivo de saída) o resultado do programa, o nó com rank 0 não difere dos outros nós em execução.

\subsubsection{Send, Receive, Broadcast}
Em um sistema baseado em passagem de mensagens, alguns comandos devem ser definidos para que estas mensagens sejam, de fato, passadas. A função \texttt{Send} envia uma quantidade de bytes para um certo nó alvo com uma \textit{tag}, sendo que estes bytes contém alguma informação útil(a mensagem) para o programa alvo. Para cada chamada de \texttt{Send} de uma fonte a um destino, deve haver uma chamada, no destino, da função \texttt{Receive} que indica a fonte da mensagem assim como o tamanho da mensagem e o tipo da mesma. Estruturas complexas(\texttt{struct}, em C) devem ser registradas em cada um dos nós que executam estas funções para que estes mantenham uma sincronia entre suas comunicações.

Já o comando \texttt{Broadcast} recebe como parâmetro uma variável, o tamanho dela, um nó fonte e um comunicador. Esta função irá então recuperar a informação contida nesta variável no nó fonte e enviá-la para todos os nós que fazem parte deste comunicador, que irão receber a informação e registrá-la na variável que eles possuem(lembrando que cada nó possui seu próprio conjunto de memória e de variáveis). Assim, é possível enviar uma informação a diversos nós de forma eficiente(sem a necessidade de múltiplas chamadas de \texttt{Send} e \texttt{Receive}).

\subsection{Distribuição de Instâncias e Execução}
A execução do algoritmo será feita em auxílio do programa \texttt{mpirun}, que recebe como parâmetros o número de nós a ser executado, os \textit{hosts}, ou máquinas, que irão executar o programa, além do próprio programa com seus próprios argumentos.

As instâncias de execução podem ser distribuídas de duas formas: por máquina ou por \textit{slot}. Na distribuição por máquina, cada máquina alvo receberá uma quantidade balanceada de instâncias, independente do hardware que a máquina executa; Já na distribuição por slot, pode-se determinar que uma máquina receba uma certa quantidade de instâncias antes de começar a criar novas instâncias na próxima máquina. Esta segunda forma de distribuição pode ser vantajosa, pois a comunicação entre nós em uma mesma máquina é mais rápida do que a comunicação entre nós de máquinas distintas(afinal, a comunicação é local), porém deve-se configurar manualmente exatamente quantos \textit{slots} uma máquina deve possuir antes de ser considerada "cheia".


\subsection{Leituras Adicionais}
O tópico de paralelização é muito mais extenso do que o escopo deste trabalho. Recomenda-se a leitura de outras fontes como \url{https://computing.llnl.gov/tutorials/parallel\_comp/#ModelsMessage/}\cite{parallelcomp:website}.


\section{Aplicando o Problema da Mochila ao Ant System}
No Problema da Mochila, soluções válidas podem ser interpretados como estados em que a mochila se encontra. Assim, uma mochila começa em um estado inicial onde suas capacidades $C_k$ estão em seu valor máximo, e nenhum item está adicionado à mochila.

A transição de estados da mochila é dada pela adição de um novo item na mesma; Ao adicionar um novo item, as restrições $C_k$ são reduzidas de acordo com os atributos do item $i$, e o valor da mochila, e portanto qualidade da solução, cresce de acordo com $v_i$(o valor de $i$). Se um item possui uma restrição $w_{ik}>C_k$, então a mochila não pode adicionar este item nela - o que significa que o estado em que a mochila se encontra não é ligado à aresta a qual este item é representado.

Levando então a consideração de que o estado da mochila é um nó, e o item é uma aresta, basta determinar uma fórmula para a atratividade $\mu$ dos itens.

Como o intuito deste trabalho é mostrar a paralelização do \textit{ACO}, não foi feito uma análise extensa sobre qual $\mu$ provém os melhores resultados em menos iterações. Usaremos então uma fórmula baseada no trabalho de Schiff, K\cite{aco:schiff}, a qual é simples de calcular:
$$
	\mu_j = \dfrac{v_j}{
		\prod r_{ij}}
$$
Onde $r_{ij}$ são todas as restrições de um item $j$ e $v_j$ é o valor do item $j$.

Dado então um número de iterações $iter$, um número de formigas $ants$ e um número de itens $n$ onde cada item possui $m$ restrições e um valor $v_j$, um valor inicial de feromônio $\varphi$ e um valor máximo de feromônio $\varphi_{\max}$, assim como capacidade máxima da mochila $C_k$, $0<k<m$, constrói-se o algoritmo final no Algoritmo \ref{acoseq}.

\begin{algorithm}[ht]
	\KwIn{\\
		$iter$: Número de iterações do sistema\\
		$ants$: Número de formigas do sistema\\
		$n$: Número de itens\\
		$m$: Número de restrições\\
		$C_k, 0<k<m$: Capacidade da mochila\\
		$v_j, 0<j<n$: Valor do item $j$\\
		$r_{ij}, 0<j<n, 0<i<m$: Restrição $i$ do item $j$\\
		$\mu_j = \dfrac{v_j}{
			\prod r_{ij}}$: Atratividade do item $j$\\
		$\varphi, \varphi_{\max}$: Valor inicial e máximo de feromônio\\
		$\rho$: Coeficiente de evaporação do feromônio\\
		$\alpha$: Peso que o feromônio tem sobre a seleção dos itens\\
		$\beta$: Peso que a atratividade tem sobre a seleção dos itens\\
	}
	
	\KwOut{$S$: estado da mochila na solução final}
	$best = \emptyset$\;
	$\forall_{0<j<n} \tau_j \gets \varphi$\;
	\While{$iter{-}{-} > 0$ }{
		$best_x\gets\emptyset \qquad \forall_{0<x<ants}$\;
		\BlankLine
		\textit{//Buscar uma solução}\;
		\For{$x\gets 0 $\textbf{ to }$ ants$}{
			$solucao\gets constroi\_solucao()$\;
			\If{($solucao > best_x$)}{$best_x\gets solucao$}
			\If{($solucao > best$)}{$best\gets solucao$}	
		}
		\BlankLine
		\textit{//Atualizar feromonio}\;
		\For{$j\gets0$ \textbf{to} $n$}{
			$\tau_j \gets \tau_j * \rho$\;
		}
		\For{$x\gets0$ \textbf{to} $ ants$}{
			\ForEach{$i$: Item $\in best_x$}{
				$\tau_i \gets \Delta\tau(best_x)$\;	
			}
		}
		
		Atualizar $\tau_j^\alpha \mu_j^\beta$ para cada item\;
	}
	return $best$\;
	\caption{Algoritmo sequencial para o ACO}
	\label{acoseq}
\end{algorithm} 

\section{Paralelização do \textit{ACO} no modelo de memória distribuída}
Uma das vantagens de um modelo de memória distribuída é que precisa-se se preocupar apenas com a informação que \textit{deve} ser distribuída entre diferentes nós do processo. No caso do \textit{ACO}, se assumirmos que todos os nós em execução possuem acesso a todos os itens do universo, e que todos os nós são capazes de construir uma solução dada um feromônio, nota-se que a única váriavel é de fato compartilhada e modificada entre as formigas é o feromônio em si: $\tau_j$ depende de cada solução encontrada por cada formiga, porém todas as outras variáveis, inclusive o composto entre Feromônio e Desejabilidade $\tau_j^\alpha \mu_j^\beta$, podem ser calculadas antes ou depois de se determinar o valor do feromônio em uma dada iteração.

\subsection{Lista de atualização de feromônio}
Quando um conjunto de formigas termina sua iteração - isto é, encontram uma solução válida e estão prontas para atualizar o feromônio - é possível criar uma lista $\Delta\tau_j \forall{j}$ tal o índice $j$ indica qual será a diferença linear do feromônio após a evaporação com o feromônio após o término da atualização. Esta lista é relativamente pequena - cada elemento pode ser dado por um par \texttt{\{int, double\}} que tem um tamanho de 12 bytes na maioria dos sistema; e uma lista de atualização iria conter apenas os itens que irão, de fato, receber um aumento de feromônio além da evaporação; Isto é especialmente útil pois no problema da mochila, apenas um subconjunto pequeno de itens é incluso na mochila em um grande universo.

\subsection{Envio da Mensagem}
Pode-se então utilizar a função \texttt{Broadcast} para enviar a lista de feromônios para todos os nós do sistema, onde cada nó possui um subconjunto de formigas(em geral, ${num\_ants}\div{num\_nos}$). Assim, cada nó irá receber as listas de todos os outros nós, onde então todos os nós irão somar as partes das listas para atualizar suas listas locais de feromônio - minimizando, assim, o tamanho e número de mensagens a serem enviadas no sistema.

Finalmente, ao término do programa(neste caso, do número de iterações), os nós podem então enviar seus melhores resultados ao nó de rank $0$ para que este possa finalizar a execução do programa com o resultado geral final.





A forma mais intuitiva de se paralelizar o \textit{ACO} é em paralelizar as formigas, ou seja, enviar as formigas em \textit{threads} separadas para que trabalhem de forma concorrente. Durante a construção de solução, não existem problemas de concorrência: variáveis globais como $\tau_i$ não são modificadas.

\subsection{Regiões críticas}
Nota-se também que o algoritmo sequencial(\ref{acoseq}) possui 4 estágios bem definidos: Construção de soluções, Evaporação de feromônio, Atualização de feromônio e Atualização da probabilidade de escolha $\tau_j^\alpha \mu_j^\beta$. Existe também uma ordem específica em que cada um destes estágios precisam ocorrer: Não se pode evaporar o feromônio antes de terminar de construir as soluções, pois estas são dependentes do feromônio. Também não se pode atualizar o feromônio antes da evaporação pois isto mudará o valor final da operação; e com certeza não se pode atualizar o feromônio \textit{durante} a evaporação. Também não se pode atualizar $\tau_j^\alpha \mu_j^\beta$ antes que o feromônio esteja completamente atualizado, e não se pode construir uma nova solução antes que $\tau_j^\alpha \mu_j^\beta$ esteja atualizado em cada item.

Isto indica que o algoritmo irá se beneficiar do uso de \textbf{barreiras}, impedindo que cada thread continue sua execução antes que todas as threads terminem o estágio específico em que se encontram. Além disso, um lock será necessário para impedir que $\tau_i$ seja modificado por várias threads ao mesmo tempo. O algoritmo \ref{acopar} mostra como ele irá funcionar para uma Thread. Para o programa final, basta verificar qual thread produziu a melhor solução.

\begin{algorithm}[ht]
		\KwIn{\\
			$N$: O numero total de nós\\
			$rank$: O identificador do nó no comunicador global, $1\leq rank \leq N$\\
			$iter$: Número de iterações do sistema\\
			$ants$: Número de formigas por nó, onde $ants * N$ dá o número total de formigas\\
			$n$: Número de itens\\
			$m$: Número de restrições\\
			$C_k, 0<k<m$: Capacidade da mochila\\
			$v_j, 0<j<n$: Valor do item $j$\\
			$r_{ij}, 0<j<n, 0<i<m$: Restrição $i$ do item $j$\\
			$\mu_j = \dfrac{v_j}{
				\prod r_{ij}}$: Atratividade do item $j$\\
			$\varphi, \varphi_{\max}$: Valor inicial e máximo de feromônio\\
			$\rho$: Coeficiente de evaporação do feromônio\\
			$\alpha$: Peso que o feromônio tem sobre a seleção dos itens\\
			$\beta$: Peso que a atratividade tem sobre a seleção dos itens\\
		}
		
		\KwOut{$S$: estado da mochila na solução final}
		
		
		$\forall_{0<j<n} \tau_j \gets \varphi$\;
		\While{$iter{-}{-} > 0$ }{
			$best_x\gets\emptyset \qquad \forall_{0<x<ants}$\;
			\BlankLine
			\For{$x\gets 1 $\textbf{ to }$ ants$}{
				$solucao\gets constroi\_solucao()$\;
				\If{($solucao > best_x$)}{$best_x\gets solucao$}
			}
			$\Delta\tau_j \gets \Delta\tau(best_x) \qquad \forall_j \in best$ \;
			$\Delta\tau' \gets \Delta\tau$
			\BlankLine
			\For{$i \gets 1 $ \textbf{to} $ N$}{
				$\Delta\phi <- \emptyset$\;
				\If{$rank = i$}{
					$\Delta\phi <- \Delta\tau'$\;	
				}
				\texttt{Broadcast}($\Delta\phi, i$, \texttt{MPI\_COMM\_WORLD})\;
				$\Delta\tau \gets \Delta\tau + \Delta\phi$\;
			}
			
			\For{$j \gets 1$ \textbf{to} $ n$}{
				$\tau_j \gets \tau_j * \rho + \Delta\tau_j$\;
				Atualizar $\tau_j^\alpha \mu_j^\beta$\;
			}
		}
		return $best$\;
		\caption{Algoritmo de uma thread para o ACO}
		\label{acopar}
\end{algorithm}

\subsection{Análise de desempenho}
O programa foi implementado na linguagem \texttt{C}, padrão \texttt{C11}, compilado com gcc versão 5.3.0 sob otimização \texttt{-O2}, e executado em um e duas máquinas com processador \textbf{i7 @ 3.5GHz}. A execução ocorreu no sistema operacional Ubuntu 14.04.

Para a coleta de dados, o programa foi executado em 1, 2, 3, 4, 8 e 16 nós, sendo testado o desempenho com uma distribuição de instâncias por máquina e por \textit{slot}, limitando cada máquina a 4 slots. Um gráfico do \textit{speedup} de cada instância pode ser verificado na figura \ref{Speedup}.

% GNUPLOT: LaTeX picture
\setlength{\unitlength}{0.240900pt}
\ifx\plotpoint\undefined\newsavebox{\plotpoint}\fi
\sbox{\plotpoint}{\rule[-0.200pt]{0.400pt}{0.400pt}}%
\begin{picture}(1500,900)(0,0)
\sbox{\plotpoint}{\rule[-0.200pt]{0.400pt}{0.400pt}}%
\put(131.0,131.0){\rule[-0.200pt]{4.818pt}{0.400pt}}
\put(111,131){\makebox(0,0)[r]{$0$}}
\put(1419.0,131.0){\rule[-0.200pt]{4.818pt}{0.400pt}}
\put(131.0,212.0){\rule[-0.200pt]{4.818pt}{0.400pt}}
\put(111,212){\makebox(0,0)[r]{$2$}}
\put(1419.0,212.0){\rule[-0.200pt]{4.818pt}{0.400pt}}
\put(131.0,292.0){\rule[-0.200pt]{4.818pt}{0.400pt}}
\put(111,292){\makebox(0,0)[r]{$4$}}
\put(1419.0,292.0){\rule[-0.200pt]{4.818pt}{0.400pt}}
\put(131.0,373.0){\rule[-0.200pt]{4.818pt}{0.400pt}}
\put(111,373){\makebox(0,0)[r]{$6$}}
\put(1419.0,373.0){\rule[-0.200pt]{4.818pt}{0.400pt}}
\put(131.0,454.0){\rule[-0.200pt]{4.818pt}{0.400pt}}
\put(111,454){\makebox(0,0)[r]{$8$}}
\put(1419.0,454.0){\rule[-0.200pt]{4.818pt}{0.400pt}}
\put(131.0,534.0){\rule[-0.200pt]{4.818pt}{0.400pt}}
\put(111,534){\makebox(0,0)[r]{$10$}}
\put(1419.0,534.0){\rule[-0.200pt]{4.818pt}{0.400pt}}
\put(131.0,615.0){\rule[-0.200pt]{4.818pt}{0.400pt}}
\put(111,615){\makebox(0,0)[r]{$12$}}
\put(1419.0,615.0){\rule[-0.200pt]{4.818pt}{0.400pt}}
\put(131.0,695.0){\rule[-0.200pt]{4.818pt}{0.400pt}}
\put(111,695){\makebox(0,0)[r]{$14$}}
\put(1419.0,695.0){\rule[-0.200pt]{4.818pt}{0.400pt}}
\put(131.0,776.0){\rule[-0.200pt]{4.818pt}{0.400pt}}
\put(111,776){\makebox(0,0)[r]{$16$}}
\put(1419.0,776.0){\rule[-0.200pt]{4.818pt}{0.400pt}}
\put(131.0,131.0){\rule[-0.200pt]{0.400pt}{4.818pt}}
\put(131,90){\makebox(0,0){$0$}}
\put(131.0,756.0){\rule[-0.200pt]{0.400pt}{4.818pt}}
\put(295.0,131.0){\rule[-0.200pt]{0.400pt}{4.818pt}}
\put(295,90){\makebox(0,0){$2$}}
\put(295.0,756.0){\rule[-0.200pt]{0.400pt}{4.818pt}}
\put(458.0,131.0){\rule[-0.200pt]{0.400pt}{4.818pt}}
\put(458,90){\makebox(0,0){$4$}}
\put(458.0,756.0){\rule[-0.200pt]{0.400pt}{4.818pt}}
\put(622.0,131.0){\rule[-0.200pt]{0.400pt}{4.818pt}}
\put(622,90){\makebox(0,0){$6$}}
\put(622.0,756.0){\rule[-0.200pt]{0.400pt}{4.818pt}}
\put(785.0,131.0){\rule[-0.200pt]{0.400pt}{4.818pt}}
\put(785,90){\makebox(0,0){$8$}}
\put(785.0,756.0){\rule[-0.200pt]{0.400pt}{4.818pt}}
\put(949.0,131.0){\rule[-0.200pt]{0.400pt}{4.818pt}}
\put(949,90){\makebox(0,0){$10$}}
\put(949.0,756.0){\rule[-0.200pt]{0.400pt}{4.818pt}}
\put(1112.0,131.0){\rule[-0.200pt]{0.400pt}{4.818pt}}
\put(1112,90){\makebox(0,0){$12$}}
\put(1112.0,756.0){\rule[-0.200pt]{0.400pt}{4.818pt}}
\put(1276.0,131.0){\rule[-0.200pt]{0.400pt}{4.818pt}}
\put(1276,90){\makebox(0,0){$14$}}
\put(1276.0,756.0){\rule[-0.200pt]{0.400pt}{4.818pt}}
\put(1439.0,131.0){\rule[-0.200pt]{0.400pt}{4.818pt}}
\put(1439,90){\makebox(0,0){$16$}}
\put(1439.0,756.0){\rule[-0.200pt]{0.400pt}{4.818pt}}
\put(131.0,131.0){\rule[-0.200pt]{0.400pt}{155.380pt}}
\put(131.0,131.0){\rule[-0.200pt]{315.097pt}{0.400pt}}
\put(1439.0,131.0){\rule[-0.200pt]{0.400pt}{155.380pt}}
\put(131.0,776.0){\rule[-0.200pt]{315.097pt}{0.400pt}}
\put(30,453){\makebox(0,0){Speedup}}
\put(785,29){\makebox(0,0){Numero de Nos}}
\put(785,838){\makebox(0,0){Tempo de Execucao}}
\sbox{\plotpoint}{\rule[-0.600pt]{1.200pt}{1.200pt}}%
\sbox{\plotpoint}{\rule[-0.200pt]{0.400pt}{0.400pt}}%
\put(1279,735){\makebox(0,0)[r]{Speedup relativo com duas máquinas}}
\sbox{\plotpoint}{\rule[-0.600pt]{1.200pt}{1.200pt}}%
\put(1299.0,735.0){\rule[-0.600pt]{24.090pt}{1.200pt}}
\put(213,171){\usebox{\plotpoint}}
\multiput(213.00,173.24)(1.077,0.500){66}{\rule{2.889pt}{0.121pt}}
\multiput(213.00,168.51)(76.003,38.000){2}{\rule{1.445pt}{1.200pt}}
\multiput(295.00,211.24)(1.086,0.500){140}{\rule{2.908pt}{0.120pt}}
\multiput(295.00,206.51)(156.964,75.000){2}{\rule{1.454pt}{1.200pt}}
\multiput(458.00,286.24)(0.869,0.500){366}{\rule{2.387pt}{0.120pt}}
\multiput(458.00,281.51)(322.045,188.000){2}{\rule{1.194pt}{1.200pt}}
\multiput(785.00,474.24)(12.374,0.500){44}{\rule{29.367pt}{0.121pt}}
\multiput(785.00,469.51)(593.048,27.000){2}{\rule{14.683pt}{1.200pt}}
\put(213,171){\makebox(0,0){$\blacktriangle$}}
\put(295,209){\makebox(0,0){$\blacktriangle$}}
\put(458,284){\makebox(0,0){$\blacktriangle$}}
\put(785,472){\makebox(0,0){$\blacktriangle$}}
\put(1439,499){\makebox(0,0){$\blacktriangle$}}
\put(1349,735){\makebox(0,0){$\blacktriangle$}}
\sbox{\plotpoint}{\rule[-0.200pt]{0.400pt}{0.400pt}}%
\put(1279,694){\makebox(0,0)[r]{Speedup relativo com uma máquina}}
\put(1299.0,694.0){\rule[-0.200pt]{24.090pt}{0.400pt}}
\put(213,171){\usebox{\plotpoint}}
\multiput(213.00,171.58)(1.003,0.498){79}{\rule{0.900pt}{0.120pt}}
\multiput(213.00,170.17)(80.132,41.000){2}{\rule{0.450pt}{0.400pt}}
\multiput(295.00,212.58)(1.099,0.498){71}{\rule{0.976pt}{0.120pt}}
\multiput(295.00,211.17)(78.975,37.000){2}{\rule{0.488pt}{0.400pt}}
\multiput(376.00,249.58)(0.641,0.499){125}{\rule{0.613pt}{0.120pt}}
\multiput(376.00,248.17)(80.729,64.000){2}{\rule{0.306pt}{0.400pt}}
\multiput(458.00,313.58)(15.401,0.492){19}{\rule{11.991pt}{0.118pt}}
\multiput(458.00,312.17)(302.112,11.000){2}{\rule{5.995pt}{0.400pt}}
\multiput(785.00,322.92)(9.713,-0.498){65}{\rule{7.794pt}{0.120pt}}
\multiput(785.00,323.17)(637.823,-34.000){2}{\rule{3.897pt}{0.400pt}}
\put(213,171){\makebox(0,0){$\bullet$}}
\put(295,212){\makebox(0,0){$\bullet$}}
\put(376,249){\makebox(0,0){$\bullet$}}
\put(458,313){\makebox(0,0){$\bullet$}}
\put(785,324){\makebox(0,0){$\bullet$}}
\put(1439,290){\makebox(0,0){$\bullet$}}
\put(1349,694){\makebox(0,0){$\bullet$}}
\put(1279,653){\makebox(0,0)[r]{Linear}}
\put(1299.0,653.0){\rule[-0.200pt]{24.090pt}{0.400pt}}
\put(213,171){\usebox{\plotpoint}}
\multiput(213.00,171.59)(1.033,0.482){9}{\rule{0.900pt}{0.116pt}}
\multiput(213.00,170.17)(10.132,6.000){2}{\rule{0.450pt}{0.400pt}}
\multiput(225.00,177.59)(0.950,0.485){11}{\rule{0.843pt}{0.117pt}}
\multiput(225.00,176.17)(11.251,7.000){2}{\rule{0.421pt}{0.400pt}}
\multiput(238.00,184.59)(1.033,0.482){9}{\rule{0.900pt}{0.116pt}}
\multiput(238.00,183.17)(10.132,6.000){2}{\rule{0.450pt}{0.400pt}}
\multiput(250.00,190.59)(1.033,0.482){9}{\rule{0.900pt}{0.116pt}}
\multiput(250.00,189.17)(10.132,6.000){2}{\rule{0.450pt}{0.400pt}}
\multiput(262.00,196.59)(1.123,0.482){9}{\rule{0.967pt}{0.116pt}}
\multiput(262.00,195.17)(10.994,6.000){2}{\rule{0.483pt}{0.400pt}}
\multiput(275.00,202.59)(1.033,0.482){9}{\rule{0.900pt}{0.116pt}}
\multiput(275.00,201.17)(10.132,6.000){2}{\rule{0.450pt}{0.400pt}}
\multiput(287.00,208.59)(1.033,0.482){9}{\rule{0.900pt}{0.116pt}}
\multiput(287.00,207.17)(10.132,6.000){2}{\rule{0.450pt}{0.400pt}}
\multiput(299.00,214.59)(1.123,0.482){9}{\rule{0.967pt}{0.116pt}}
\multiput(299.00,213.17)(10.994,6.000){2}{\rule{0.483pt}{0.400pt}}
\multiput(312.00,220.59)(1.033,0.482){9}{\rule{0.900pt}{0.116pt}}
\multiput(312.00,219.17)(10.132,6.000){2}{\rule{0.450pt}{0.400pt}}
\multiput(324.00,226.59)(1.123,0.482){9}{\rule{0.967pt}{0.116pt}}
\multiput(324.00,225.17)(10.994,6.000){2}{\rule{0.483pt}{0.400pt}}
\multiput(337.00,232.59)(0.874,0.485){11}{\rule{0.786pt}{0.117pt}}
\multiput(337.00,231.17)(10.369,7.000){2}{\rule{0.393pt}{0.400pt}}
\multiput(349.00,239.59)(1.033,0.482){9}{\rule{0.900pt}{0.116pt}}
\multiput(349.00,238.17)(10.132,6.000){2}{\rule{0.450pt}{0.400pt}}
\multiput(361.00,245.59)(1.123,0.482){9}{\rule{0.967pt}{0.116pt}}
\multiput(361.00,244.17)(10.994,6.000){2}{\rule{0.483pt}{0.400pt}}
\multiput(374.00,251.59)(1.033,0.482){9}{\rule{0.900pt}{0.116pt}}
\multiput(374.00,250.17)(10.132,6.000){2}{\rule{0.450pt}{0.400pt}}
\multiput(386.00,257.59)(1.123,0.482){9}{\rule{0.967pt}{0.116pt}}
\multiput(386.00,256.17)(10.994,6.000){2}{\rule{0.483pt}{0.400pt}}
\multiput(399.00,263.59)(1.033,0.482){9}{\rule{0.900pt}{0.116pt}}
\multiput(399.00,262.17)(10.132,6.000){2}{\rule{0.450pt}{0.400pt}}
\multiput(411.00,269.59)(1.033,0.482){9}{\rule{0.900pt}{0.116pt}}
\multiput(411.00,268.17)(10.132,6.000){2}{\rule{0.450pt}{0.400pt}}
\multiput(423.00,275.59)(1.123,0.482){9}{\rule{0.967pt}{0.116pt}}
\multiput(423.00,274.17)(10.994,6.000){2}{\rule{0.483pt}{0.400pt}}
\multiput(436.00,281.59)(1.033,0.482){9}{\rule{0.900pt}{0.116pt}}
\multiput(436.00,280.17)(10.132,6.000){2}{\rule{0.450pt}{0.400pt}}
\multiput(448.00,287.59)(1.033,0.482){9}{\rule{0.900pt}{0.116pt}}
\multiput(448.00,286.17)(10.132,6.000){2}{\rule{0.450pt}{0.400pt}}
\multiput(460.00,293.59)(0.950,0.485){11}{\rule{0.843pt}{0.117pt}}
\multiput(460.00,292.17)(11.251,7.000){2}{\rule{0.421pt}{0.400pt}}
\multiput(473.00,300.59)(1.033,0.482){9}{\rule{0.900pt}{0.116pt}}
\multiput(473.00,299.17)(10.132,6.000){2}{\rule{0.450pt}{0.400pt}}
\multiput(485.00,306.59)(1.123,0.482){9}{\rule{0.967pt}{0.116pt}}
\multiput(485.00,305.17)(10.994,6.000){2}{\rule{0.483pt}{0.400pt}}
\multiput(498.00,312.59)(1.033,0.482){9}{\rule{0.900pt}{0.116pt}}
\multiput(498.00,311.17)(10.132,6.000){2}{\rule{0.450pt}{0.400pt}}
\multiput(510.00,318.59)(1.033,0.482){9}{\rule{0.900pt}{0.116pt}}
\multiput(510.00,317.17)(10.132,6.000){2}{\rule{0.450pt}{0.400pt}}
\multiput(522.00,324.59)(1.123,0.482){9}{\rule{0.967pt}{0.116pt}}
\multiput(522.00,323.17)(10.994,6.000){2}{\rule{0.483pt}{0.400pt}}
\multiput(535.00,330.59)(1.033,0.482){9}{\rule{0.900pt}{0.116pt}}
\multiput(535.00,329.17)(10.132,6.000){2}{\rule{0.450pt}{0.400pt}}
\multiput(547.00,336.59)(1.123,0.482){9}{\rule{0.967pt}{0.116pt}}
\multiput(547.00,335.17)(10.994,6.000){2}{\rule{0.483pt}{0.400pt}}
\multiput(560.00,342.59)(1.033,0.482){9}{\rule{0.900pt}{0.116pt}}
\multiput(560.00,341.17)(10.132,6.000){2}{\rule{0.450pt}{0.400pt}}
\multiput(572.00,348.59)(0.874,0.485){11}{\rule{0.786pt}{0.117pt}}
\multiput(572.00,347.17)(10.369,7.000){2}{\rule{0.393pt}{0.400pt}}
\multiput(584.00,355.59)(1.123,0.482){9}{\rule{0.967pt}{0.116pt}}
\multiput(584.00,354.17)(10.994,6.000){2}{\rule{0.483pt}{0.400pt}}
\multiput(597.00,361.59)(1.033,0.482){9}{\rule{0.900pt}{0.116pt}}
\multiput(597.00,360.17)(10.132,6.000){2}{\rule{0.450pt}{0.400pt}}
\multiput(609.00,367.59)(1.123,0.482){9}{\rule{0.967pt}{0.116pt}}
\multiput(609.00,366.17)(10.994,6.000){2}{\rule{0.483pt}{0.400pt}}
\multiput(622.00,373.59)(1.033,0.482){9}{\rule{0.900pt}{0.116pt}}
\multiput(622.00,372.17)(10.132,6.000){2}{\rule{0.450pt}{0.400pt}}
\multiput(634.00,379.59)(1.033,0.482){9}{\rule{0.900pt}{0.116pt}}
\multiput(634.00,378.17)(10.132,6.000){2}{\rule{0.450pt}{0.400pt}}
\multiput(646.00,385.59)(1.123,0.482){9}{\rule{0.967pt}{0.116pt}}
\multiput(646.00,384.17)(10.994,6.000){2}{\rule{0.483pt}{0.400pt}}
\multiput(659.00,391.59)(1.033,0.482){9}{\rule{0.900pt}{0.116pt}}
\multiput(659.00,390.17)(10.132,6.000){2}{\rule{0.450pt}{0.400pt}}
\multiput(671.00,397.59)(1.033,0.482){9}{\rule{0.900pt}{0.116pt}}
\multiput(671.00,396.17)(10.132,6.000){2}{\rule{0.450pt}{0.400pt}}
\multiput(683.00,403.59)(0.950,0.485){11}{\rule{0.843pt}{0.117pt}}
\multiput(683.00,402.17)(11.251,7.000){2}{\rule{0.421pt}{0.400pt}}
\multiput(696.00,410.59)(1.033,0.482){9}{\rule{0.900pt}{0.116pt}}
\multiput(696.00,409.17)(10.132,6.000){2}{\rule{0.450pt}{0.400pt}}
\multiput(708.00,416.59)(1.123,0.482){9}{\rule{0.967pt}{0.116pt}}
\multiput(708.00,415.17)(10.994,6.000){2}{\rule{0.483pt}{0.400pt}}
\multiput(721.00,422.59)(1.033,0.482){9}{\rule{0.900pt}{0.116pt}}
\multiput(721.00,421.17)(10.132,6.000){2}{\rule{0.450pt}{0.400pt}}
\multiput(733.00,428.59)(1.033,0.482){9}{\rule{0.900pt}{0.116pt}}
\multiput(733.00,427.17)(10.132,6.000){2}{\rule{0.450pt}{0.400pt}}
\multiput(745.00,434.59)(1.123,0.482){9}{\rule{0.967pt}{0.116pt}}
\multiput(745.00,433.17)(10.994,6.000){2}{\rule{0.483pt}{0.400pt}}
\multiput(758.00,440.59)(1.033,0.482){9}{\rule{0.900pt}{0.116pt}}
\multiput(758.00,439.17)(10.132,6.000){2}{\rule{0.450pt}{0.400pt}}
\multiput(770.00,446.59)(1.123,0.482){9}{\rule{0.967pt}{0.116pt}}
\multiput(770.00,445.17)(10.994,6.000){2}{\rule{0.483pt}{0.400pt}}
\multiput(783.00,452.59)(1.033,0.482){9}{\rule{0.900pt}{0.116pt}}
\multiput(783.00,451.17)(10.132,6.000){2}{\rule{0.450pt}{0.400pt}}
\multiput(795.00,458.59)(1.033,0.482){9}{\rule{0.900pt}{0.116pt}}
\multiput(795.00,457.17)(10.132,6.000){2}{\rule{0.450pt}{0.400pt}}
\multiput(807.00,464.59)(0.950,0.485){11}{\rule{0.843pt}{0.117pt}}
\multiput(807.00,463.17)(11.251,7.000){2}{\rule{0.421pt}{0.400pt}}
\multiput(820.00,471.59)(1.033,0.482){9}{\rule{0.900pt}{0.116pt}}
\multiput(820.00,470.17)(10.132,6.000){2}{\rule{0.450pt}{0.400pt}}
\multiput(832.00,477.59)(1.033,0.482){9}{\rule{0.900pt}{0.116pt}}
\multiput(832.00,476.17)(10.132,6.000){2}{\rule{0.450pt}{0.400pt}}
\multiput(844.00,483.59)(1.123,0.482){9}{\rule{0.967pt}{0.116pt}}
\multiput(844.00,482.17)(10.994,6.000){2}{\rule{0.483pt}{0.400pt}}
\multiput(857.00,489.59)(1.033,0.482){9}{\rule{0.900pt}{0.116pt}}
\multiput(857.00,488.17)(10.132,6.000){2}{\rule{0.450pt}{0.400pt}}
\multiput(869.00,495.59)(1.123,0.482){9}{\rule{0.967pt}{0.116pt}}
\multiput(869.00,494.17)(10.994,6.000){2}{\rule{0.483pt}{0.400pt}}
\multiput(882.00,501.59)(1.033,0.482){9}{\rule{0.900pt}{0.116pt}}
\multiput(882.00,500.17)(10.132,6.000){2}{\rule{0.450pt}{0.400pt}}
\multiput(894.00,507.59)(1.033,0.482){9}{\rule{0.900pt}{0.116pt}}
\multiput(894.00,506.17)(10.132,6.000){2}{\rule{0.450pt}{0.400pt}}
\multiput(906.00,513.59)(1.123,0.482){9}{\rule{0.967pt}{0.116pt}}
\multiput(906.00,512.17)(10.994,6.000){2}{\rule{0.483pt}{0.400pt}}
\multiput(919.00,519.59)(0.874,0.485){11}{\rule{0.786pt}{0.117pt}}
\multiput(919.00,518.17)(10.369,7.000){2}{\rule{0.393pt}{0.400pt}}
\multiput(931.00,526.59)(1.123,0.482){9}{\rule{0.967pt}{0.116pt}}
\multiput(931.00,525.17)(10.994,6.000){2}{\rule{0.483pt}{0.400pt}}
\multiput(944.00,532.59)(1.033,0.482){9}{\rule{0.900pt}{0.116pt}}
\multiput(944.00,531.17)(10.132,6.000){2}{\rule{0.450pt}{0.400pt}}
\multiput(956.00,538.59)(1.033,0.482){9}{\rule{0.900pt}{0.116pt}}
\multiput(956.00,537.17)(10.132,6.000){2}{\rule{0.450pt}{0.400pt}}
\multiput(968.00,544.59)(1.123,0.482){9}{\rule{0.967pt}{0.116pt}}
\multiput(968.00,543.17)(10.994,6.000){2}{\rule{0.483pt}{0.400pt}}
\multiput(981.00,550.59)(1.033,0.482){9}{\rule{0.900pt}{0.116pt}}
\multiput(981.00,549.17)(10.132,6.000){2}{\rule{0.450pt}{0.400pt}}
\multiput(993.00,556.59)(1.033,0.482){9}{\rule{0.900pt}{0.116pt}}
\multiput(993.00,555.17)(10.132,6.000){2}{\rule{0.450pt}{0.400pt}}
\multiput(1005.00,562.59)(1.123,0.482){9}{\rule{0.967pt}{0.116pt}}
\multiput(1005.00,561.17)(10.994,6.000){2}{\rule{0.483pt}{0.400pt}}
\multiput(1018.00,568.59)(1.033,0.482){9}{\rule{0.900pt}{0.116pt}}
\multiput(1018.00,567.17)(10.132,6.000){2}{\rule{0.450pt}{0.400pt}}
\multiput(1030.00,574.59)(0.950,0.485){11}{\rule{0.843pt}{0.117pt}}
\multiput(1030.00,573.17)(11.251,7.000){2}{\rule{0.421pt}{0.400pt}}
\multiput(1043.00,581.59)(1.033,0.482){9}{\rule{0.900pt}{0.116pt}}
\multiput(1043.00,580.17)(10.132,6.000){2}{\rule{0.450pt}{0.400pt}}
\multiput(1055.00,587.59)(1.033,0.482){9}{\rule{0.900pt}{0.116pt}}
\multiput(1055.00,586.17)(10.132,6.000){2}{\rule{0.450pt}{0.400pt}}
\multiput(1067.00,593.59)(1.123,0.482){9}{\rule{0.967pt}{0.116pt}}
\multiput(1067.00,592.17)(10.994,6.000){2}{\rule{0.483pt}{0.400pt}}
\multiput(1080.00,599.59)(1.033,0.482){9}{\rule{0.900pt}{0.116pt}}
\multiput(1080.00,598.17)(10.132,6.000){2}{\rule{0.450pt}{0.400pt}}
\multiput(1092.00,605.59)(1.123,0.482){9}{\rule{0.967pt}{0.116pt}}
\multiput(1092.00,604.17)(10.994,6.000){2}{\rule{0.483pt}{0.400pt}}
\multiput(1105.00,611.59)(1.033,0.482){9}{\rule{0.900pt}{0.116pt}}
\multiput(1105.00,610.17)(10.132,6.000){2}{\rule{0.450pt}{0.400pt}}
\multiput(1117.00,617.59)(1.033,0.482){9}{\rule{0.900pt}{0.116pt}}
\multiput(1117.00,616.17)(10.132,6.000){2}{\rule{0.450pt}{0.400pt}}
\multiput(1129.00,623.59)(1.123,0.482){9}{\rule{0.967pt}{0.116pt}}
\multiput(1129.00,622.17)(10.994,6.000){2}{\rule{0.483pt}{0.400pt}}
\multiput(1142.00,629.59)(0.874,0.485){11}{\rule{0.786pt}{0.117pt}}
\multiput(1142.00,628.17)(10.369,7.000){2}{\rule{0.393pt}{0.400pt}}
\multiput(1154.00,636.59)(1.123,0.482){9}{\rule{0.967pt}{0.116pt}}
\multiput(1154.00,635.17)(10.994,6.000){2}{\rule{0.483pt}{0.400pt}}
\multiput(1167.00,642.59)(1.033,0.482){9}{\rule{0.900pt}{0.116pt}}
\multiput(1167.00,641.17)(10.132,6.000){2}{\rule{0.450pt}{0.400pt}}
\multiput(1179.00,648.59)(1.033,0.482){9}{\rule{0.900pt}{0.116pt}}
\multiput(1179.00,647.17)(10.132,6.000){2}{\rule{0.450pt}{0.400pt}}
\multiput(1191.00,654.59)(1.123,0.482){9}{\rule{0.967pt}{0.116pt}}
\multiput(1191.00,653.17)(10.994,6.000){2}{\rule{0.483pt}{0.400pt}}
\multiput(1204.00,660.59)(1.033,0.482){9}{\rule{0.900pt}{0.116pt}}
\multiput(1204.00,659.17)(10.132,6.000){2}{\rule{0.450pt}{0.400pt}}
\multiput(1216.00,666.59)(1.033,0.482){9}{\rule{0.900pt}{0.116pt}}
\multiput(1216.00,665.17)(10.132,6.000){2}{\rule{0.450pt}{0.400pt}}
\multiput(1228.00,672.59)(1.123,0.482){9}{\rule{0.967pt}{0.116pt}}
\multiput(1228.00,671.17)(10.994,6.000){2}{\rule{0.483pt}{0.400pt}}
\multiput(1241.00,678.59)(1.033,0.482){9}{\rule{0.900pt}{0.116pt}}
\multiput(1241.00,677.17)(10.132,6.000){2}{\rule{0.450pt}{0.400pt}}
\multiput(1253.00,684.59)(1.123,0.482){9}{\rule{0.967pt}{0.116pt}}
\multiput(1253.00,683.17)(10.994,6.000){2}{\rule{0.483pt}{0.400pt}}
\multiput(1266.00,690.59)(0.874,0.485){11}{\rule{0.786pt}{0.117pt}}
\multiput(1266.00,689.17)(10.369,7.000){2}{\rule{0.393pt}{0.400pt}}
\multiput(1278.00,697.59)(1.033,0.482){9}{\rule{0.900pt}{0.116pt}}
\multiput(1278.00,696.17)(10.132,6.000){2}{\rule{0.450pt}{0.400pt}}
\multiput(1290.00,703.59)(1.123,0.482){9}{\rule{0.967pt}{0.116pt}}
\multiput(1290.00,702.17)(10.994,6.000){2}{\rule{0.483pt}{0.400pt}}
\multiput(1303.00,709.59)(1.033,0.482){9}{\rule{0.900pt}{0.116pt}}
\multiput(1303.00,708.17)(10.132,6.000){2}{\rule{0.450pt}{0.400pt}}
\multiput(1315.00,715.59)(1.123,0.482){9}{\rule{0.967pt}{0.116pt}}
\multiput(1315.00,714.17)(10.994,6.000){2}{\rule{0.483pt}{0.400pt}}
\multiput(1328.00,721.59)(1.033,0.482){9}{\rule{0.900pt}{0.116pt}}
\multiput(1328.00,720.17)(10.132,6.000){2}{\rule{0.450pt}{0.400pt}}
\multiput(1340.00,727.59)(1.033,0.482){9}{\rule{0.900pt}{0.116pt}}
\multiput(1340.00,726.17)(10.132,6.000){2}{\rule{0.450pt}{0.400pt}}
\multiput(1352.00,733.59)(1.123,0.482){9}{\rule{0.967pt}{0.116pt}}
\multiput(1352.00,732.17)(10.994,6.000){2}{\rule{0.483pt}{0.400pt}}
\multiput(1365.00,739.59)(1.033,0.482){9}{\rule{0.900pt}{0.116pt}}
\multiput(1365.00,738.17)(10.132,6.000){2}{\rule{0.450pt}{0.400pt}}
\multiput(1377.00,745.59)(0.874,0.485){11}{\rule{0.786pt}{0.117pt}}
\multiput(1377.00,744.17)(10.369,7.000){2}{\rule{0.393pt}{0.400pt}}
\multiput(1389.00,752.59)(1.123,0.482){9}{\rule{0.967pt}{0.116pt}}
\multiput(1389.00,751.17)(10.994,6.000){2}{\rule{0.483pt}{0.400pt}}
\multiput(1402.00,758.59)(1.033,0.482){9}{\rule{0.900pt}{0.116pt}}
\multiput(1402.00,757.17)(10.132,6.000){2}{\rule{0.450pt}{0.400pt}}
\multiput(1414.00,764.59)(1.123,0.482){9}{\rule{0.967pt}{0.116pt}}
\multiput(1414.00,763.17)(10.994,6.000){2}{\rule{0.483pt}{0.400pt}}
\multiput(1427.00,770.59)(1.033,0.482){9}{\rule{0.900pt}{0.116pt}}
\multiput(1427.00,769.17)(10.132,6.000){2}{\rule{0.450pt}{0.400pt}}
\put(131.0,131.0){\rule[-0.200pt]{0.400pt}{155.380pt}}
\put(131.0,131.0){\rule[-0.200pt]{315.097pt}{0.400pt}}
\put(1439.0,131.0){\rule[-0.200pt]{0.400pt}{155.380pt}}
\put(131.0,776.0){\rule[-0.200pt]{315.097pt}{0.400pt}}
\end{picture}


Imediatamente nota-se que existe uma \textit{Superlinearidade} no \textit{speedup} dos programas em uma instância em particular: Quando o número de nós é igual ao número total de núcleos físicos de processamento. A hipótese deste fenômeno ter ocorrido se dá ao modo em que o processador i7 se comporta quando sobre uma grande carga de processos ou com o modo em que o sistema operacional escalona os processos quando nesta instância em específica. Testes de outras instâncias não registrados mostram um padrão neste fenômeno em que o speedup superlinear é diretamente relacionado a este número de processos(4 por máquina).

Nota-se também o fenômeno mais esperado de que, ao passar do número de processadores físicos, ocorre uma queda na eficiência de processamento pois o sistema estará passando uma grande quantidade de tempo escalonando os processos - ocorrendo um slowdown perceptível quando o número de processos começa a ficar muito maior do que o número de núcleos de processamento.

Atribui-se o alto desempenho ao fato de que as mensagens passadas são pequenas e rapidamente processadas entre nós; Além disso, os computadores se encontravam fisicamente próximos e é esperado que uma rede mais congestionada ou esparsa, ou mesmo um aumento no número de máquinas, causaria uma perda de eficácia em relação às instâncias testadas. Mesmo assim, o ganho de tempo no processamento distribuído para uma instância suficientemente grande sobrepõe qualquer perda de eficácia gerada por uma rede esparsa.


A implementação do projeto está disponível em \url{https://github.com/Barbiero/KnapsackAntSystem/tree/mpibranch}.

\section{Conclusões}
A paralelização do \textit{ACO} pode ser feita com o uso de barreiras; o desempenho da paralelização é completamente dependente de como o problema é modelado e como a memória é gerenciada, assim como em como o problema toma vantagem da arquitetura interna do computador que irá executar o programa.

\bibliographystyle{ieeetr}
\bibliography{ref}

\end{document}




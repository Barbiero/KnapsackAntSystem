\documentclass[12pt,a4paper]{article}

\usepackage[utf8]{inputenc}
\usepackage[brazilian,english]{babel}
\usepackage{amsmath}
\usepackage{amssymb}
\usepackage[hidelinks]{hyperref}
\usepackage[ruled,vlined,algosection]{algorithm2e}
\AtBeginDocument{\let\textlabel\label}


\begin{document}

\selectlanguage{brazilian}
\title{Paralelização do Algoritmo de Colonização de Formigas com o Problema da Mochila de Múltiplas Restrições}

\author{Pedro Barbiero\\
	Universidade Estadual de Maringá\\
	{\tt pedrobarbiero@gmail.com}
}

\maketitle

\selectlanguage{english}
\begin{abstract}
	This report describes the process of parallelizing the Ant Colonization Optimization algorithm, used to solve the Multiple-Constraint Knapsack problem, through a distributed memory model. It was noted that, given a sufficiently small message size that the overhead from this model is negligible, achieving very high speedup factors. It was also noted that at certain instances, there was a super-linear speedup and it was hipotesized that this happens due to the hyper-threading technology provided on the Intel i7 processor, but no conclusion was reached from this phenomenom.
	
\end{abstract}

\selectlanguage{brazilian}

\section{Introdução}
	O algoritmo de otimização da colônia de formigas(sigla em inglês, {\it ACO}) é uma heurística baseada em probabilidade no qual se baseia no conceito de formigas que buscam alimentos e deixam trilhas de feromônio para que outras formigas encontrem este mesmo alimento\cite{antsystem}. Em termos de computacionais, considera-se que as \textit{formigas} são elementos que irão atravessar um grafo de um problema, enquanto a \textit{comida} será uma solução válida para este problema, tal que as formigas irão encontrar uma solução boa(e possívelmente, ótima) para o dado problema.
	
	O problema em questão é nomeado \textit{Problema da Mochila de Múltiplas Restrições}, em inglês \textit{Multiple Knapsack Problem}. Este problema é uma variação do problema da mochila onde cada nó do grafo do problema representa um estado da mochila, e cada aresta representa a introdução de um novo item na mochila. Este problema foi mostrado como NP-Completo por Gens, G\cite{complexity}.
	
	Neste trabalho, iremos utilizar o \textit{ACO} para resolver o problema em questão, e então melhorar o tempo de solução do problema criando múltiplas instâncias do programa de solução, tais que elas se comuniquem e resolvam o problema da mesma forma que em um programa sequencial, podendo ser escalada para diversas máquinas ou mesmo em diversos processadores de uma única máquina.

\section{Problema da Mochila de Múltiplas Restrições}

\subsection{O Problema da Mochila}
O Problema da Mochila tradicional é um problema de otimização combinatória, normalmente descrito como  \textbf{"Dado um conjunto de possíveis itens, cada um com um certo valor e um certo peso, e uma mochila com uma capacidade de peso limite, qual é a melhor combinação de itens que a mochila consegue suportar, tal que o valor seja maximizado?"}.

Matematicamente, o problema é descrito como:
	\begin{displaymath}
	\max \sum_{i=1}^{n} v_i x_i
	\end{displaymath}
	\begin{displaymath}
	tal\ que \sum_{i=1}^{n} w_i x_i \leq W
	\end{displaymath}
	Onde 
	\begin{itemize}
		\item $n$ é a quantidade de itens que podem ser postos na mochila,
		\item $v_i$ é o valor do item $i$,
		\item  $w_i$ é o peso do item $i$, 
		\item $W$ é a capacidade da mochila e
		\item $x_i$ é a quantidade de vezes em que o item $i$ é colocado na mochila.
	\end{itemize}
	
Em particular, a variação mais comum do problema a ser encontrado é a \textit{Mochila Binária}, onde $x_i \in \{0,1\}$, ou seja, existe apenas um item de cada tipo que pode ser posto na mochila, e este item não pode ser fracionado.

\subsection{Mochila Binária com Múltiplas Restrições}
Outra variação do problema da mochila binária determina que a mochila possui mais do que uma restrição - por exemplo, além da capacidade de peso existe um limite de volume. O problema é então descrito na forma
\begin{eqnarray}
\max & \sum_{i=1}^{m} \sum_{j=1}^{n} p_{ij} x_{ij} & \nonumber\\
tal\ que & \sum_{j=1}^{n} w_j x_{ij} \leq W_i, & \forall\ 1 \leq i \leq m,\nonumber\\
		& \sum_{i=1}^{m} x_{ij} \leq 1, & \forall\ 1 \leq j \leq n, \nonumber\\
		& x_{ij} \in \{0,1\}, & (\forall\ 1 \leq j \leq n )\wedge (\forall\ 1 \leq i \leq m) \nonumber
\end{eqnarray}




\section{Sistema de Colonização de Formigas}
O Algoritmo de Colonização de Formigas, ou \textit{ACO}, é uma meta-heurística de otimização aplicada em diversas áreas e é útil quando um tempo curto é necessário para resolver um problema \cite{acomkp}.  É um sistema onde múltiplos agentes(formigas) agem de uma forma relativamente simples porém interagem entre si(através de um feromônio) resultando em um comportamento complexo.

O ACO consiste em repetir diversas vezes um conjunto de operações:
\begin{enumerate}
	\item Buscar solução baseado no feromônio atual
	\item Atualizar o Feromônio
	\item (opcional) Operações de \textit{Daemon}
\end{enumerate}

Estas operações são repetidas várias vezes até que uma condição de parada seja satisfeita: um limite no número de iterações, tempo ou mesmo uma interrupção externa. Em geral, espera-se que o algoritmo encontre uma melhor solução se este ciclo for repetido mais vezes.

\subsection{Construção da Solução}
Como é comum para o ACO tratar o problema como um grafo, a construção da solução pode ser feita pelo Algoritmo \ref{buildSol}.

\begin{algorithm}[hb]
	$Solucao \gets \emptyset$\;

	\While{true}{
		$A \gets $ conjunto de arestas que levam a uma solução válida\;
		\If{$A = \emptyset $}{
			$return$ \textbf{Solucao}\;
		}
			
		Solucao avança em uma aresta de A\;
	}
	\caption{Algoritmo de construcão de solução}
	\label{buildSol}
\end{algorithm}

Embora o algoritmo pareça relativamente simples, o método de seleção da aresta que será avançada é o diferencial do \textit{ACO}. Neste sistema, atribuem-se a cada aresta dois valores chamados \textit{feromônio}($\tau$) e \textit{atratividade}($\mu$). Estes valores podem, ou não, ser relativos ao estado atual da solução no momento de sua construção, e relativos aos atributos das arestas; A seleção é feita então probabilisticamente: Dado um conjunto $A_i$ de possíveis arestas para construir a solução em um passo $i$, a chance de selecionar uma aresta $j$ é dada por
\[ 
	p_j = 
	\begin{cases}
	\dfrac{\tau_j^\alpha \mu_j^\beta}
	{\sum_{k \in A_i} \tau_k^\alpha \mu_k^\beta},
	 & \text{se } j \in A_i,\\
	 0 & \text{se } j \notin A_i.
	\end{cases}
\]
O valor de $\mu_j$ é dado por uma combinação de fatores do problema; no problema da mochila, por exemplo, pode se referir a uma relação entre os atributos de um item e a capacidade total da mochila. Este valor deve ser determinado como uma heurística para ajudar a selecionar as melhores arestas para a solução. As constantes $\alpha$ e $\beta$ determinam o peso que o feromônio $\tau$ e a atratividade $\mu$ terão, respectivamente.

\subsection{Atualização do Feromônio}
Em cada iteração do sistema, o feromônio será atualizado de acordo com a melhor solução encontrada por uma formiga nesta iteração, assim aumentando a probabilidade de seguir um caminho de uma boa solução para a próxima iteração. Além de existir a soma do feromônio, ele também irá \textit{evaporar}, isto é, o valor numérico do feromônio será multiplicado por uma constante $0<\rho<1$. Assim, a atualização do feromônio é dada pelo algoritmo \ref{pherUpdate}.

\begin{algorithm}[ht]
	\KwIn{$S_k$: Conjunto de soluções encontradas por todas as formigas\\
		$\rho$: Taxa de evaporação do feromônio, $0\leq\rho\leq1$}
	
	\ForEach{$j :$ aresta no sistema}{
		$\tau_j \gets \tau_j * \rho$\;
	}
	
	\ForEach{$S\in S_k$}{
		\ForEach{$i$: Aresta $\in S$}{
			$\tau_i \gets \tau_i * \Delta\tau(S)$\;
		}	
	}
	\caption{Atualização de Ferômonio}
	\label{pherUpdate}
\end{algorithm}

Note que $\Delta\tau$ é um valor que pode, ou não, depender de atributos da solução S, mas é comum que este valor seja relativo à qualidade da solução encontrada - assim, soluções melhores colocam mais feromônio no sistema.

\subsection{Operações de Daemon}
São operações dependentes da implementação. Por exemplo, o cálculo de $\tau_j^\alpha \mu_j^\beta$ é computacionalmente custoso, e pode ser útil atualizá-lo em cada aresta apenas uma vez por iteração ao invés de calculá-lo em todas as chamadas, assim otimizando a velocidade de execução do programa.




\section{Paralelização de Algoritmos}
Algoritmos sequenciais podem tomar vantagem das arquiteturas atuais de processadores e executar suas operações de forma concorrente com a ajuda de certas bibliotecas de programação. Neste trabalho foi-se utilizada a bilbioteca \texttt{pthreads.h} na linguagem {\tt C}. Nesta seção, serão explicados alguns mecanismos que são necessários para paralelizar um algoritmo que normalmente seria sequencial.

\subsection{Threads}
Em um fluxo de execução comum, o \textit{processo} irá chamar uma função inicial(normalmente {\tt main}) e então seguir o código sequencialmente a partir da chamada da função até que seu fluxo de execução seja finalizado - via uma chamada de retorno da função {\tt main}, chamada da função {\tt exit} ou mesmo por uma interrupção externa. Entretanto, utilizando-se de uma biblioteca de paralelização e chamadas específicas ao sistema operacional, é possível criar \textit{threads}, que são fluxos de execução que não são necessariamente executados em sequencia. Estas threads irão então seguir seu fluxo de execução independentes entre si, aproveitando as arquiteturas de multi-processadores para que o código ocorra em paralelo.

\subsection{Memória compartilhada e condiçoes de corrida}
Quando se trata de um código paralelo, a memória pode - ou não - ser compartilhada, dependendo da implementação do mesmo. Com {\tt pthreads}, a memória do \textit{stack} do programa é compartilhada entre si, o que permite a diferentes Threads se comunicarem ou acessarem informações globais entre si. Existe, porém um problema quando a memória entre threads é compartilhada: condições de corrida(\textit{race conditions}).

Quando duas threads tentam acessar e manipular uma mesma variável ao mesmo tempo, condições de corrida podem ocorrer. Suponha que duas threads, $T_0$ e $T_1$, tentem executar a operação $a_{++}$. Esta operação pode ser expandida desta forma:
\begin{align}
	i \gets& a \nonumber\\
	i \gets& i+1 \nonumber\\
	a \gets& i \nonumber
\end{align}

Onde $i$ é um registrador do processador. Se, por exemplo, $T_0$ e $T_1$ executam estas instruções ao mesmo tempo, quando chegarem na terceira linha $a \gets i$, a variável $a$ receberá o valor $a+1$, mesmo que duas Threads tenham executado a operação de incremento. Para que isto não ocorra, torna-se necessário o uso de instrumentos de controle de concorrência, chamados de \textbf{locks}.

\subsection{Locks}
\textbf{Lock} ou \textbf{Mutually exclusive lock}(\texttt{mutex\_lock}) é uma instrução que determina uma \textit{região crítica} para que duas threads não possam acessar esta região crítica ao mesmo tempo. Assim, no exemplo acima, o operador $a_{++}$ poderia ser implementado como
\begin{align}
lock()\quad&\nonumber\\
i\gets& a\nonumber\\
i\gets& i+1\nonumber\\
a\gets& i\nonumber\\
unlock()&\nonumber
\end{align}
Neste exemplo, se $T_0$ começar a executar a instrução, ele irá adquirir a trava da região crítica, enquanto $T_1$ irá pausar sua execução na função \texttt{lock()}, até que $T_0$ libere a trava com a função \texttt{unlock()} e permita que $T_1$ adquiria a trava da região crítica, o que irá impedir outra thread de acessar esta região até que $T_1$ a libere.

No padrão \texttt{C11}, a \textit{keyword} \texttt{\_Atomic} determina que uma variável é uma região crítica, e quaisquer operações feitas sobre esta variável são implicitamente atômicas - efetivamente agindo como se existissem \textit{locks} sobre estas variáveis. Com a biblioteca \texttt{pthreads}, o uso de funções como \texttt{pthread\_mutex\_lock} e \texttt{pthread\_mutex\_unlock} permite um controle maior de regiões críticas, não se limitando apenas a operações de uma variável.

\subsection{Barreiras}
Além dos locks, outra forma de controle de concorrência é a barreira. Barreiras são estruturas que determinam um ponto do programa em que uma thread deve esperar até que $n$ threads atinjam este ponto, onde $n$ é um número especificado para a barreira, normalmente sendo igual ao número de Threads que são relacionadas àquele ponto em específico.

Barreiras podem ser implementadas de diferentes formas; no programa anexado a este trabalho, uma forma particular de barreira chamada \textbf{Tree Barrier} foi implementada, onde a a barreira é organizada na forma de uma árvore, em que cada thread está em um nó-folha da árvore e quando esta thread chega no ponto da barreira, ela irá informar o nó pai. Quando o pai das folhas verifica que todas as folhas informaram que a barreira terminou, este irá informar o seu pai até que a raiz note que todas as folhas tenham chegado no ponto de execução desejado, então liberando-as automaticamente.

\subsection{Leituras Adicionais}
O tópico de paralelização é muito mais extenso do que o escopo deste trabalho. Recomenda-se a leitura de outras fontes como \url{https://computing.llnl.gov/tutorials/parallel\_comp/}\cite{parallelcomp:website}.


\section{Aplicando o Problema da Mochila ao Ant System}
No Problema da Mochila, soluções válidas podem ser interpretados como estados em que a mochila se encontra. Assim, uma mochila começa em um estado inicial onde suas capacidades $C_k$ estão em seu valor máximo, e nenhum item está adicionado à mochila.

A transição de estados da mochila é dada pela adição de um novo item na mesma; Ao adicionar um novo item, as restrições $C_k$ são reduzidas de acordo com os atributos do item $i$, e o valor da mochila, e portanto qualidade da solução, cresce de acordo com $v_i$(o valor de $i$). Se um item possui uma restrição $w_{ik}>C_k$, então a mochila não pode adicionar este item nela - o que significa que o estado em que a mochila se encontra não é ligado à aresta a qual este item é representado.

Levando então a consideração de que o estado da mochila é um nó, e o item é uma aresta, basta determinar uma fórmula para a atratividade $\mu$ dos itens.

Como o intuito deste trabalho é mostrar a paralelização do \textit{ACO}, não foi feito uma análise extensa sobre qual $\mu$ provém os melhores resultados em menos iterações. Usaremos então uma fórmula baseada no trabalho de Schiff, K\cite{aco:schiff}, a qual é simples de calcular:
$$
	\mu_j = \dfrac{v_j}{
		\prod r_{ij}}
$$
Onde $r_{ij}$ são todas as restrições de um item $j$ e $v_j$ é o valor do item $j$.

Dado então um número de iterações $iter$, um número de formigas $ants$ e um número de itens $n$ onde cada item possui $m$ restrições e um valor $v_j$, um valor inicial de feromônio $\varphi$ e um valor máximo de feromônio $\varphi_{\max}$, assim como capacidade máxima da mochila $C_k$, $0<k<m$, constrói-se o algoritmo final no Algoritmo \ref{acoseq}.

\begin{algorithm}[ht]
	\KwIn{\\
		$iter$: Número de iterações do sistema\\
		$ants$: Número de formigas do sistema\\
		$n$: Número de itens\\
		$m$: Número de restrições\\
		$C_k, 0<k<m$: Capacidade da mochila\\
		$v_j, 0<j<n$: Valor do item $j$\\
		$r_{ij}, 0<j<n, 0<i<m$: Restrição $i$ do item $j$\\
		$\mu_j = \dfrac{v_j}{
			\prod r_{ij}}$: Atratividade do item $j$\\
		$\varphi, \varphi_{\max}$: Valor inicial e máximo de feromônio\\
		$\rho$: Coeficiente de evaporação do feromônio\\
		$\alpha$: Peso que o feromônio tem sobre a seleção dos itens\\
		$\beta$: Peso que a atratividade tem sobre a seleção dos itens\\
	}
	
	\KwOut{$S$: estado da mochila na solução final}
	$best = \emptyset$\;
	$\forall_{0<j<n} \tau_j \gets \varphi$\;
	\While{$iter{-}{-} > 0$ }{
		$best_x\gets\emptyset \qquad \forall_{0<x<ants}$\;
		\BlankLine
		\textit{//Buscar uma solução}\;
		\For{$x\gets 0 $\textbf{ to }$ ants$}{
			$solucao\gets constroi\_solucao()$\;
			\If{($solucao > best_x$)}{$best_x\gets solucao$}
			\If{($solucao > best$)}{$best\gets solucao$}	
		}
		\BlankLine
		\textit{//Atualizar feromonio}\;
		\For{$j\gets0$ \textbf{to} $n$}{
			$\tau_j \gets \tau_j * \rho$\;
		}
		\For{$x\gets0$ \textbf{to} $ ants$}{
			\ForEach{$i$: Item $\in best_x$}{
				$\tau_i \gets \Delta\tau(best_x)$\;	
			}
		}
		
		Atualizar $\tau_j^\alpha \mu_j^\beta$ para cada item\;
	}
	return $best$\;
	\caption{Algoritmo sequencial para o ACO}
	\label{acoseq}
\end{algorithm} 

\section{Paralelização do \textit{ACO}}
A forma mais intuitiva de se paralelizar o \textit{ACO} é em paralelizar as formigas, ou seja, enviar as formigas em \textit{threads} separadas para que trabalhem de forma concorrente. Durante a construção de solução, não existem problemas de concorrência: variáveis globais como $\tau_i$ não são modificadas.

\subsection{Regiões críticas}
Nota-se também que o algoritmo sequencial(\ref{acoseq}) possui 4 estágios bem definidos: Construção de soluções, Evaporação de feromônio, Atualização de feromônio e Atualização da probabilidade de escolha $\tau_j^\alpha \mu_j^\beta$. Existe também uma ordem específica em que cada um destes estágios precisam ocorrer: Não se pode evaporar o feromônio antes de terminar de construir as soluções, pois estas são dependentes do feromônio. Também não se pode atualizar o feromônio antes da evaporação pois isto mudará o valor final da operação; e com certeza não se pode atualizar o feromônio \textit{durante} a evaporação. Também não se pode atualizar $\tau_j^\alpha \mu_j^\beta$ antes que o feromônio esteja completamente atualizado, e não se pode construir uma nova solução antes que $\tau_j^\alpha \mu_j^\beta$ esteja atualizado em cada item.

Isto indica que o algoritmo irá se beneficiar do uso de \textbf{barreiras}, impedindo que cada thread continue sua execução antes que todas as threads terminem o estágio específico em que se encontram. Além disso, um lock será necessário para impedir que $\tau_i$ seja modificado por várias threads ao mesmo tempo. O algoritmo \ref{acopar} mostra como ele irá funcionar para uma Thread. Para o programa final, basta verificar qual thread produziu a melhor solução.

\begin{algorithm}[ht]
		\KwIn{\\
			$barreira$: Estrura de barreira inicializada antes da criação das Threads\\
			$T$: O numero total de threads\\
			$threadid$: O identificador da thread, $0\leq threadid \leq T$\\
			$iter$: Número de iterações do sistema\\
			$ants$: Número de formigas do sistema\\
			$n$: Número de itens\\
			$m$: Número de restrições\\
			$C_k, 0<k<m$: Capacidade da mochila\\
			$v_j, 0<j<n$: Valor do item $j$\\
			$r_{ij}, 0<j<n, 0<i<m$: Restrição $i$ do item $j$\\
			$\mu_j = \dfrac{v_j}{
				\prod r_{ij}}$: Atratividade do item $j$\\
			$\varphi, \varphi_{\max}$: Valor inicial e máximo de feromônio\\
			$\rho$: Coeficiente de evaporação do feromônio\\
			$\alpha$: Peso que o feromônio tem sobre a seleção dos itens\\
			$\beta$: Peso que a atratividade tem sobre a seleção dos itens\\
		}
		
		\KwOut{$S$: estado da mochila na solução final}
		
		$ini_{ant} \gets threadid * \lceil\frac{ants}{T}\rceil$\;
		$fin_{ant} \gets \min(ini_{ant} + \lceil\frac{ants}{T}\rceil, ants)$\;
		$ini_{item} \gets threadid * \lceil\frac{n}{T}\rceil$\;
		$fin_{item} \gets \min(ini_{item} + \lceil\frac{n}{T}\rceil, n)$\;
		
		$\forall_{0<j<n} \tau_j \gets \varphi$\;
		\While{$iter{-}{-} > 0$ }{
			$best_x\gets\emptyset \qquad \forall_{0<x<ants}$\;
			\BlankLine
			\texttt{buscar\_solucao($best_x, best$)(\ref{algbuscaSol})}\;
			barreira\_esperar($barreira$)\;
			\BlankLine
			\texttt{evap\_feromonio()(\ref{algevapPher})}\;
			barreira\_esperar($barreira$)\;
			\BlankLine
			\texttt{atualiza\_feromonio($best_x$)(\ref{algupdPher})}\;
			barreira\_esperar($barreira$)\;
			
			\For{$j\gets ini_{item}$ \textbf{to} $ fin_{item}$}{
				Atualizar $\tau_j^\alpha \mu_j^\beta$\;
			}
			barreira\_esperar($barreira$)\;
		}
		return $best$\;
		\caption{Algoritmo de uma thread para o ACO}
		\label{acopar}
\end{algorithm}

\begin{algorithm}[htb]
	\For{$x\gets ini_{ant} $\textbf{ to }$ fin_{ant}$}{
		$solucao\gets constroi\_solucao()$\;
		\If{($solucao > best_x$)}{$best_x\gets solucao$}
		\If{($solucao > best$)}{$best\gets solucao$}	
	}
	\caption{Busca de solução no algoritmo paralelo}
	\label{algbuscaSol}
\end{algorithm}
\begin{algorithm}[htb]
	\For{$j\gets ini_{item}$ \textbf{to} $ fin_{item}$}{
		$\tau_j \gets \tau_j * \rho$\;
	}
	\caption{Evaporar Feromonios no algoritmo paralelo}
	\label{algevapPher}
\end{algorithm}
\begin{algorithm}[htb]
	\For{$x\gets ini_{ant}$ \textbf{to} $ fin_{ant}$}{
		\ForEach{$i$: Item $\in best_x$}{
			\texttt{lock($\tau_i$)}\;
			$\tau_i \gets \Delta\tau(best_x)$\;	
			\texttt{unlock($\tau_i$)}\;
		}
	}
	\caption{Atualizar feromonios no algoritmo paralelo}
	\label{algupdPher}
\end{algorithm}

\subsection{Seleção de Formigas}
Dado um número qualquer de formigas, pode-se distribui-las entre as threads quase-igualmente(com uma exceção para a última thread, caso $ants\mod T \neq 0$) com duas operações:
\begin{align}
ini_{ant} \gets &threadid * \lceil\frac{ants}{T}\rceil\nonumber\\
fin_{ant} \gets &\min(ini_{ant} + \lceil\frac{ants}{T}\rceil, ants)\nonumber
\end{align}
Assim, enquanto um programa que atinge todas as formigas irá iterar sobre as formigas $[0,ants)$, as threads irão iterar igualmente sobre $[ini_{ant}, fin_{ant})$. Além disso, as operações que ocorrem sobre os $n$ itens são analogamente distribuídas entre as threads, onde cada thread irá processar os itens $[ini_{item}, fin_{item})$

\subsection{Análise de desempenho}
O programa foi implementado na linguagem \texttt{C}, padrão \texttt{C11}, compilado com gcc versão 5.3.0 sob otimização \texttt{-O2}, e executado em um notebook \textbf{Dell Inspiron 14z 5423}, que possui um processador \textbf{i5 3317U 1.7GHz}. Este processador possui dois núcleos físicos e dois virtuais, totalizando quatro núcleos de processamento, com um cache de 128KB/512KB/3072KB para L1/L2/L3 respectivamente. A execução ocorreu no sistema operacional Elementary OS "freya", baseado no Ubuntu 14.04.

Para a coleta de dados de desempenho, o programa foi executado utilizando 1, 2, 4 e 8 threads cada um executados 10 vezes, e mais 10 vezes com uma versão não paralela do programa(ou seja, sem a inclusão da biblioteca \texttt{pthreads} ou qualquer mecanismo de paralelismo). O tempo reportado foi coletado com um cronometro implementado dentro do programa, e não inclui o tempo de processamento não relevante à execução do algoritmo(por exemplo, leitura de dados de entrada). A tabela \ref{tab:tempos} mostra o resultado da coleta de dados.

\begin{table} [ht]
	\centering
	\caption{Tempo de execução médio (10 execuções)}
	\label{tab:tempos}
	\begin{tabular}{|c|c|c|c|c|}
		\hline
		\textbf{Nº de } & {\bf Média } & 
		{\bf Desvio Padrão} & {\bf Speedup}& \textbf{Speedup}\\
		Threads & de tempo & ($\sigma$) & (vs sequencial) & (vs 1 thread)\\
		\hline
		Sequencial & 4'42"114 & 12"070 & 1 & --\\
		\hline
		1 & 6'3"115 & 1"853 & 0.777 & 1 \\
		\hline
		2 & 3'23"765 & 10"734 & 1.260 & 1.628 \\
		\hline
		4 & 3'16"29 & 7"888 & 1.439 & 1.852 \\
		\hline
		8 & 4'11"818 & 1"579 & 1.120 & 1.442\\
		\hline
	\end{tabular}
\end{table}

Nota-se que existe uma perda de desempenho relativo ao aumentar o número de Threads; Idealmente, a coluna de \textit{Speedup} deveria se aproximar ao número de threads. Entretanto, no caso de teste existem alguns problemas que causam uma perda de desempenho. Primeiramente, em uma arquitetura com apenas 4 núcleos de processamento, é natural que, ao passar desse número de threads, exista uma perda de processamento. Isto é visualizado com o fato de 8 threads serem mais lentas do que 4 threads.

Além disso, existe uma grande perda de desempenho com o \textit{overhead} do paralelismo - isto é verificado com a diferença de tempo entre o programa Sequencial e de 1 Thread. Em particular, a existência de um lock para cada $\tau_j$ causa um overhead notável.

Em seguida, nota-se uma perda de desempenho entre 2 e 4 threads. Enquanto um Speedup de 1.628 para 2 threads é \textit{aceitável}(porém possivelmente melhorável, se o código implementado), o speedup de apenas 1.852 para quatro threads é extremamente baixo - esta perda de desempenho é uma consequencia do \textit{overhead} em conjunto com o fato de que o computador utilizado possui dois núcleos \textit{virtuais}, que dividem memória com seus núcleos físicos - e com a divisão de memória, existe um número maior de erros de cache.

\subsection{Possíveis otimizações}
O programa foi escrito com suporte a um número dinâmico de itens e de restrições. Em \texttt{C}, isto significa que não é possível alinhar precisamente os dados dos itens com os dados de suas restrições: como estes não são definidos a tempo de compilação, é necessário alocá-los em tempo de execução o que implica que a memória das restrições $r_{ij}$ não ficará alinhada corretamente à estrutura do item $i$; Isto significa que todas as operações sobre um item que leva em consideração suas restrições causarão uma grande quantidade de misses de cache.

Outra possível otimização seria considerar o feromônio como uma matriz individual a cada thread. Isto significaria que não seria necessário correr os problemas de corrida para atualizar o feromônio, e que cada thread agiria como uma colônia de formigas independente.

A implementação do projeto está disponível em \url{https://github.com/Barbiero/KnapsackAntSystem}.

\section{Conclusões}
A paralelização do \textit{ACO} pode ser feita com o uso de barreiras; o desempenho da paralelização é completamente dependente de como o problema é modelado e como a memória é gerenciada, assim como em como o problema toma vantagem da arquitetura interna do computador que irá executar o programa.

\bibliographystyle{ieeetr}
\bibliography{ref}

\end{document}




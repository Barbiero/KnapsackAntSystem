\section{Aplicando o Problema da Mochila ao Ant System}
No Problema da Mochila, soluções válidas podem ser interpretados como estados em que a mochila se encontra. Assim, uma mochila começa em um estado inicial onde suas capacidades $C_k$ estão em seu valor máximo, e nenhum item está adicionado à mochila.

A transição de estados da mochila é dada pela adição de um novo item na mesma; Ao adicionar um novo item, as restrições $C_k$ são reduzidas de acordo com os atributos do item $i$, e o valor da mochila, e portanto qualidade da solução, cresce de acordo com $v_i$(o valor de $i$). Se um item possui uma restrição $w_{ik}>C_k$, então a mochila não pode adicionar este item nela - o que significa que o estado em que a mochila se encontra não é ligado à aresta a qual este item é representado.

Levando então a consideração de que o estado da mochila é um nó, e o item é uma aresta, basta determinar uma fórmula para a atratividade $\mu$ dos itens.

Como o intuito deste trabalho é mostrar a paralelização do \textit{ACO}, não foi feito uma análise extensa sobre qual $\mu$ provém os melhores resultados em menos iterações. Usaremos então uma fórmula baseada no trabalho de Schiff, K\cite{aco:schiff}, a qual é simples de calcular:
$$
	\mu_j = \dfrac{v_j}{
		\prod r_{ij}}
$$
Onde $r_{ij}$ são todas as restrições de um item $j$ e $v_j$ é o valor do item $j$.

Dado então um número de iterações $iter$, um número de formigas $ants$ e um número de itens $n$ onde cada item possui $m$ restrições e um valor $v_j$, um valor inicial de feromônio $\varphi$ e um valor máximo de feromônio $\varphi_{\max}$, assim como capacidade máxima da mochila $C_k$, $0<k<m$, constrói-se o algoritmo final no Algoritmo \ref{acoseq}.

\begin{algorithm}[ht]
	\KwIn{\\
		$iter$: Número de iterações do sistema\\
		$ants$: Número de formigas do sistema\\
		$n$: Número de itens\\
		$m$: Número de restrições\\
		$C_k, 0<k<m$: Capacidade da mochila\\
		$v_j, 0<j<n$: Valor do item $j$\\
		$r_{ij}, 0<j<n, 0<i<m$: Restrição $i$ do item $j$\\
		$\mu_j = \dfrac{v_j}{
			\prod r_{ij}}$: Atratividade do item $j$\\
		$\varphi, \varphi_{\max}$: Valor inicial e máximo de feromônio\\
		$\rho$: Coeficiente de evaporação do feromônio\\
		$\alpha$: Peso que o feromônio tem sobre a seleção dos itens\\
		$\beta$: Peso que a atratividade tem sobre a seleção dos itens\\
	}
	
	\KwOut{$S$: estado da mochila na solução final}
	$best = \emptyset$\;
	$\forall_{0<j<n} \tau_j \gets \varphi$\;
	\While{$iter{-}{-} > 0$ }{
		$best_x\gets\emptyset \qquad \forall_{0<x<ants}$\;
		\BlankLine
		\textit{//Buscar uma solução}\;
		\For{$x\gets 0 $\textbf{ to }$ ants$}{
			$solucao\gets constroi\_solucao()$\;
			\If{($solucao > best_x$)}{$best_x\gets solucao$}
			\If{($solucao > best$)}{$best\gets solucao$}	
		}
		\BlankLine
		\textit{//Atualizar feromonio}\;
		\For{$j\gets0$ \textbf{to} $n$}{
			$\tau_j \gets \tau_j * \rho$\;
		}
		\For{$x\gets0$ \textbf{to} $ ants$}{
			\ForEach{$i$: Item $\in best_x$}{
				$\tau_i \gets \Delta\tau(best_x)$\;	
			}
		}
		
		Atualizar $\tau_j^\alpha \mu_j^\beta$ para cada item\;
	}
	return $best$\;
	\caption{Algoritmo sequencial para o ACO}
	\label{acoseq}
\end{algorithm} 

\section{Paralelização do \textit{ACO} no modelo de memória distribuída}
Uma das vantagens de um modelo de memória distribuída é que precisa-se se preocupar apenas com a informação que \textit{deve} ser distribuída entre diferentes nós do processo. No caso do \textit{ACO}, se assumirmos que todos os nós em execução possuem acesso a todos os itens do universo, e que todos os nós são capazes de construir uma solução dada um feromônio, nota-se que a única váriavel é de fato compartilhada e modificada entre as formigas é o feromônio em si: $\tau_j$ depende de cada solução encontrada por cada formiga, porém todas as outras variáveis, inclusive o composto entre Feromônio e Desejabilidade $\tau_j^\alpha \mu_j^\beta$, podem ser calculadas antes ou depois de se determinar o valor do feromônio em uma dada iteração.

\subsection{Lista de atualização de feromônio}
Quando um conjunto de formigas termina sua iteração - isto é, encontram uma solução válida e estão prontas para atualizar o feromônio - é possível criar uma lista $\Delta\tau_j \forall{j}$ tal o índice $j$ indica qual será a diferença linear do feromônio após a evaporação com o feromônio após o término da atualização. Esta lista é relativamente pequena - cada elemento pode ser dado por um par \texttt{\{int, double\}} que tem um tamanho de 12 bytes na maioria dos sistema; e uma lista de atualização iria conter apenas os itens que irão, de fato, receber um aumento de feromônio além da evaporação; Isto é especialmente útil pois no problema da mochila, apenas um subconjunto pequeno de itens é incluso na mochila em um grande universo.

\subsection{Envio da Mensagem}
Pode-se então utilizar a função \texttt{Broadcast} para enviar a lista de feromônios para todos os nós do sistema, onde cada nó possui um subconjunto de formigas(em geral, ${num\_ants}\div{num\_nos}$). Assim, cada nó irá receber as listas de todos os outros nós, onde então todos os nós irão somar as partes das listas para atualizar suas listas locais de feromônio - minimizando, assim, o tamanho e número de mensagens a serem enviadas no sistema.

Finalmente, ao término do programa(neste caso, do número de iterações), os nós podem então enviar seus melhores resultados ao nó de rank $0$ para que este possa finalizar a execução do programa com o resultado geral final.





A forma mais intuitiva de se paralelizar o \textit{ACO} é em paralelizar as formigas, ou seja, enviar as formigas em \textit{threads} separadas para que trabalhem de forma concorrente. Durante a construção de solução, não existem problemas de concorrência: variáveis globais como $\tau_i$ não são modificadas.

\subsection{Regiões críticas}
Nota-se também que o algoritmo sequencial(\ref{acoseq}) possui 4 estágios bem definidos: Construção de soluções, Evaporação de feromônio, Atualização de feromônio e Atualização da probabilidade de escolha $\tau_j^\alpha \mu_j^\beta$. Existe também uma ordem específica em que cada um destes estágios precisam ocorrer: Não se pode evaporar o feromônio antes de terminar de construir as soluções, pois estas são dependentes do feromônio. Também não se pode atualizar o feromônio antes da evaporação pois isto mudará o valor final da operação; e com certeza não se pode atualizar o feromônio \textit{durante} a evaporação. Também não se pode atualizar $\tau_j^\alpha \mu_j^\beta$ antes que o feromônio esteja completamente atualizado, e não se pode construir uma nova solução antes que $\tau_j^\alpha \mu_j^\beta$ esteja atualizado em cada item.

Isto indica que o algoritmo irá se beneficiar do uso de \textbf{barreiras}, impedindo que cada thread continue sua execução antes que todas as threads terminem o estágio específico em que se encontram. Além disso, um lock será necessário para impedir que $\tau_i$ seja modificado por várias threads ao mesmo tempo. O algoritmo \ref{acopar} mostra como ele irá funcionar para uma Thread. Para o programa final, basta verificar qual thread produziu a melhor solução.

\begin{algorithm}[ht]
		\KwIn{\\
			$N$: O numero total de nós\\
			$rank$: O identificador do nó no comunicador global, $1\leq rank \leq N$\\
			$iter$: Número de iterações do sistema\\
			$ants$: Número de formigas por nó, onde $ants * N$ dá o número total de formigas\\
			$n$: Número de itens\\
			$m$: Número de restrições\\
			$C_k, 0<k<m$: Capacidade da mochila\\
			$v_j, 0<j<n$: Valor do item $j$\\
			$r_{ij}, 0<j<n, 0<i<m$: Restrição $i$ do item $j$\\
			$\mu_j = \dfrac{v_j}{
				\prod r_{ij}}$: Atratividade do item $j$\\
			$\varphi, \varphi_{\max}$: Valor inicial e máximo de feromônio\\
			$\rho$: Coeficiente de evaporação do feromônio\\
			$\alpha$: Peso que o feromônio tem sobre a seleção dos itens\\
			$\beta$: Peso que a atratividade tem sobre a seleção dos itens\\
		}
		
		\KwOut{$S$: estado da mochila na solução final}
		
		
		$\forall_{0<j<n} \tau_j \gets \varphi$\;
		\While{$iter{-}{-} > 0$ }{
			$best_x\gets\emptyset \qquad \forall_{0<x<ants}$\;
			\BlankLine
			\For{$x\gets 1 $\textbf{ to }$ ants$}{
				$solucao\gets constroi\_solucao()$\;
				\If{($solucao > best_x$)}{$best_x\gets solucao$}
			}
			$\Delta\tau_j \gets \Delta\tau(best_x) \qquad \forall_j \in best$ \;
			$\Delta\tau' \gets \Delta\tau$
			\BlankLine
			\For{$i \gets 1 $ \textbf{to} $ N$}{
				$\Delta\phi <- \emptyset$\;
				\If{$rank = i$}{
					$\Delta\phi <- \Delta\tau'$\;	
				}
				\texttt{Broadcast}($\Delta\phi, i$, \texttt{MPI\_COMM\_WORLD})\;
				$\Delta\tau \gets \Delta\tau + \Delta\phi$\;
			}
			
			\For{$j \gets 1$ \textbf{to} $ n$}{
				$\tau_j \gets \tau_j * \rho + \Delta\tau_j$\;
				Atualizar $\tau_j^\alpha \mu_j^\beta$\;
			}
		}
		return $best$\;
		\caption{Algoritmo de uma thread para o ACO}
		\label{acopar}
\end{algorithm}

\subsection{Análise de desempenho}
O programa foi implementado na linguagem \texttt{C}, padrão \texttt{C11}, compilado com gcc versão 5.3.0 sob otimização \texttt{-O2}, e executado em um e duas máquinas com processador \textbf{i7 @ 3.5GHz}. A execução ocorreu no sistema operacional Ubuntu 14.04.

Para a coleta de dados, o programa foi executado em 1, 2, 3, 4, 8 e 16 nós, sendo testado o desempenho com uma distribuição de instâncias por máquina e por \textit{slot}, limitando cada máquina a 4 slots. Um gráfico do \textit{speedup} de cada instância pode ser verificado na figura \ref{Speedup}.

% GNUPLOT: LaTeX picture
\setlength{\unitlength}{0.240900pt}
\ifx\plotpoint\undefined\newsavebox{\plotpoint}\fi
\sbox{\plotpoint}{\rule[-0.200pt]{0.400pt}{0.400pt}}%
\begin{picture}(1500,900)(0,0)
\sbox{\plotpoint}{\rule[-0.200pt]{0.400pt}{0.400pt}}%
\put(131.0,131.0){\rule[-0.200pt]{4.818pt}{0.400pt}}
\put(111,131){\makebox(0,0)[r]{$0$}}
\put(1419.0,131.0){\rule[-0.200pt]{4.818pt}{0.400pt}}
\put(131.0,212.0){\rule[-0.200pt]{4.818pt}{0.400pt}}
\put(111,212){\makebox(0,0)[r]{$2$}}
\put(1419.0,212.0){\rule[-0.200pt]{4.818pt}{0.400pt}}
\put(131.0,292.0){\rule[-0.200pt]{4.818pt}{0.400pt}}
\put(111,292){\makebox(0,0)[r]{$4$}}
\put(1419.0,292.0){\rule[-0.200pt]{4.818pt}{0.400pt}}
\put(131.0,373.0){\rule[-0.200pt]{4.818pt}{0.400pt}}
\put(111,373){\makebox(0,0)[r]{$6$}}
\put(1419.0,373.0){\rule[-0.200pt]{4.818pt}{0.400pt}}
\put(131.0,454.0){\rule[-0.200pt]{4.818pt}{0.400pt}}
\put(111,454){\makebox(0,0)[r]{$8$}}
\put(1419.0,454.0){\rule[-0.200pt]{4.818pt}{0.400pt}}
\put(131.0,534.0){\rule[-0.200pt]{4.818pt}{0.400pt}}
\put(111,534){\makebox(0,0)[r]{$10$}}
\put(1419.0,534.0){\rule[-0.200pt]{4.818pt}{0.400pt}}
\put(131.0,615.0){\rule[-0.200pt]{4.818pt}{0.400pt}}
\put(111,615){\makebox(0,0)[r]{$12$}}
\put(1419.0,615.0){\rule[-0.200pt]{4.818pt}{0.400pt}}
\put(131.0,695.0){\rule[-0.200pt]{4.818pt}{0.400pt}}
\put(111,695){\makebox(0,0)[r]{$14$}}
\put(1419.0,695.0){\rule[-0.200pt]{4.818pt}{0.400pt}}
\put(131.0,776.0){\rule[-0.200pt]{4.818pt}{0.400pt}}
\put(111,776){\makebox(0,0)[r]{$16$}}
\put(1419.0,776.0){\rule[-0.200pt]{4.818pt}{0.400pt}}
\put(131.0,131.0){\rule[-0.200pt]{0.400pt}{4.818pt}}
\put(131,90){\makebox(0,0){$0$}}
\put(131.0,756.0){\rule[-0.200pt]{0.400pt}{4.818pt}}
\put(295.0,131.0){\rule[-0.200pt]{0.400pt}{4.818pt}}
\put(295,90){\makebox(0,0){$2$}}
\put(295.0,756.0){\rule[-0.200pt]{0.400pt}{4.818pt}}
\put(458.0,131.0){\rule[-0.200pt]{0.400pt}{4.818pt}}
\put(458,90){\makebox(0,0){$4$}}
\put(458.0,756.0){\rule[-0.200pt]{0.400pt}{4.818pt}}
\put(622.0,131.0){\rule[-0.200pt]{0.400pt}{4.818pt}}
\put(622,90){\makebox(0,0){$6$}}
\put(622.0,756.0){\rule[-0.200pt]{0.400pt}{4.818pt}}
\put(785.0,131.0){\rule[-0.200pt]{0.400pt}{4.818pt}}
\put(785,90){\makebox(0,0){$8$}}
\put(785.0,756.0){\rule[-0.200pt]{0.400pt}{4.818pt}}
\put(949.0,131.0){\rule[-0.200pt]{0.400pt}{4.818pt}}
\put(949,90){\makebox(0,0){$10$}}
\put(949.0,756.0){\rule[-0.200pt]{0.400pt}{4.818pt}}
\put(1112.0,131.0){\rule[-0.200pt]{0.400pt}{4.818pt}}
\put(1112,90){\makebox(0,0){$12$}}
\put(1112.0,756.0){\rule[-0.200pt]{0.400pt}{4.818pt}}
\put(1276.0,131.0){\rule[-0.200pt]{0.400pt}{4.818pt}}
\put(1276,90){\makebox(0,0){$14$}}
\put(1276.0,756.0){\rule[-0.200pt]{0.400pt}{4.818pt}}
\put(1439.0,131.0){\rule[-0.200pt]{0.400pt}{4.818pt}}
\put(1439,90){\makebox(0,0){$16$}}
\put(1439.0,756.0){\rule[-0.200pt]{0.400pt}{4.818pt}}
\put(131.0,131.0){\rule[-0.200pt]{0.400pt}{155.380pt}}
\put(131.0,131.0){\rule[-0.200pt]{315.097pt}{0.400pt}}
\put(1439.0,131.0){\rule[-0.200pt]{0.400pt}{155.380pt}}
\put(131.0,776.0){\rule[-0.200pt]{315.097pt}{0.400pt}}
\put(30,453){\makebox(0,0){Speedup}}
\put(785,29){\makebox(0,0){Numero de Nos}}
\put(785,838){\makebox(0,0){Tempo de Execucao}}
\sbox{\plotpoint}{\rule[-0.600pt]{1.200pt}{1.200pt}}%
\sbox{\plotpoint}{\rule[-0.200pt]{0.400pt}{0.400pt}}%
\put(1279,735){\makebox(0,0)[r]{Speedup relativo com duas máquinas}}
\sbox{\plotpoint}{\rule[-0.600pt]{1.200pt}{1.200pt}}%
\put(1299.0,735.0){\rule[-0.600pt]{24.090pt}{1.200pt}}
\put(213,171){\usebox{\plotpoint}}
\multiput(213.00,173.24)(1.077,0.500){66}{\rule{2.889pt}{0.121pt}}
\multiput(213.00,168.51)(76.003,38.000){2}{\rule{1.445pt}{1.200pt}}
\multiput(295.00,211.24)(1.086,0.500){140}{\rule{2.908pt}{0.120pt}}
\multiput(295.00,206.51)(156.964,75.000){2}{\rule{1.454pt}{1.200pt}}
\multiput(458.00,286.24)(0.869,0.500){366}{\rule{2.387pt}{0.120pt}}
\multiput(458.00,281.51)(322.045,188.000){2}{\rule{1.194pt}{1.200pt}}
\multiput(785.00,474.24)(12.374,0.500){44}{\rule{29.367pt}{0.121pt}}
\multiput(785.00,469.51)(593.048,27.000){2}{\rule{14.683pt}{1.200pt}}
\put(213,171){\makebox(0,0){$\blacktriangle$}}
\put(295,209){\makebox(0,0){$\blacktriangle$}}
\put(458,284){\makebox(0,0){$\blacktriangle$}}
\put(785,472){\makebox(0,0){$\blacktriangle$}}
\put(1439,499){\makebox(0,0){$\blacktriangle$}}
\put(1349,735){\makebox(0,0){$\blacktriangle$}}
\sbox{\plotpoint}{\rule[-0.200pt]{0.400pt}{0.400pt}}%
\put(1279,694){\makebox(0,0)[r]{Speedup relativo com uma máquina}}
\put(1299.0,694.0){\rule[-0.200pt]{24.090pt}{0.400pt}}
\put(213,171){\usebox{\plotpoint}}
\multiput(213.00,171.58)(1.003,0.498){79}{\rule{0.900pt}{0.120pt}}
\multiput(213.00,170.17)(80.132,41.000){2}{\rule{0.450pt}{0.400pt}}
\multiput(295.00,212.58)(1.099,0.498){71}{\rule{0.976pt}{0.120pt}}
\multiput(295.00,211.17)(78.975,37.000){2}{\rule{0.488pt}{0.400pt}}
\multiput(376.00,249.58)(0.641,0.499){125}{\rule{0.613pt}{0.120pt}}
\multiput(376.00,248.17)(80.729,64.000){2}{\rule{0.306pt}{0.400pt}}
\multiput(458.00,313.58)(15.401,0.492){19}{\rule{11.991pt}{0.118pt}}
\multiput(458.00,312.17)(302.112,11.000){2}{\rule{5.995pt}{0.400pt}}
\multiput(785.00,322.92)(9.713,-0.498){65}{\rule{7.794pt}{0.120pt}}
\multiput(785.00,323.17)(637.823,-34.000){2}{\rule{3.897pt}{0.400pt}}
\put(213,171){\makebox(0,0){$\bullet$}}
\put(295,212){\makebox(0,0){$\bullet$}}
\put(376,249){\makebox(0,0){$\bullet$}}
\put(458,313){\makebox(0,0){$\bullet$}}
\put(785,324){\makebox(0,0){$\bullet$}}
\put(1439,290){\makebox(0,0){$\bullet$}}
\put(1349,694){\makebox(0,0){$\bullet$}}
\put(1279,653){\makebox(0,0)[r]{Linear}}
\put(1299.0,653.0){\rule[-0.200pt]{24.090pt}{0.400pt}}
\put(213,171){\usebox{\plotpoint}}
\multiput(213.00,171.59)(1.033,0.482){9}{\rule{0.900pt}{0.116pt}}
\multiput(213.00,170.17)(10.132,6.000){2}{\rule{0.450pt}{0.400pt}}
\multiput(225.00,177.59)(0.950,0.485){11}{\rule{0.843pt}{0.117pt}}
\multiput(225.00,176.17)(11.251,7.000){2}{\rule{0.421pt}{0.400pt}}
\multiput(238.00,184.59)(1.033,0.482){9}{\rule{0.900pt}{0.116pt}}
\multiput(238.00,183.17)(10.132,6.000){2}{\rule{0.450pt}{0.400pt}}
\multiput(250.00,190.59)(1.033,0.482){9}{\rule{0.900pt}{0.116pt}}
\multiput(250.00,189.17)(10.132,6.000){2}{\rule{0.450pt}{0.400pt}}
\multiput(262.00,196.59)(1.123,0.482){9}{\rule{0.967pt}{0.116pt}}
\multiput(262.00,195.17)(10.994,6.000){2}{\rule{0.483pt}{0.400pt}}
\multiput(275.00,202.59)(1.033,0.482){9}{\rule{0.900pt}{0.116pt}}
\multiput(275.00,201.17)(10.132,6.000){2}{\rule{0.450pt}{0.400pt}}
\multiput(287.00,208.59)(1.033,0.482){9}{\rule{0.900pt}{0.116pt}}
\multiput(287.00,207.17)(10.132,6.000){2}{\rule{0.450pt}{0.400pt}}
\multiput(299.00,214.59)(1.123,0.482){9}{\rule{0.967pt}{0.116pt}}
\multiput(299.00,213.17)(10.994,6.000){2}{\rule{0.483pt}{0.400pt}}
\multiput(312.00,220.59)(1.033,0.482){9}{\rule{0.900pt}{0.116pt}}
\multiput(312.00,219.17)(10.132,6.000){2}{\rule{0.450pt}{0.400pt}}
\multiput(324.00,226.59)(1.123,0.482){9}{\rule{0.967pt}{0.116pt}}
\multiput(324.00,225.17)(10.994,6.000){2}{\rule{0.483pt}{0.400pt}}
\multiput(337.00,232.59)(0.874,0.485){11}{\rule{0.786pt}{0.117pt}}
\multiput(337.00,231.17)(10.369,7.000){2}{\rule{0.393pt}{0.400pt}}
\multiput(349.00,239.59)(1.033,0.482){9}{\rule{0.900pt}{0.116pt}}
\multiput(349.00,238.17)(10.132,6.000){2}{\rule{0.450pt}{0.400pt}}
\multiput(361.00,245.59)(1.123,0.482){9}{\rule{0.967pt}{0.116pt}}
\multiput(361.00,244.17)(10.994,6.000){2}{\rule{0.483pt}{0.400pt}}
\multiput(374.00,251.59)(1.033,0.482){9}{\rule{0.900pt}{0.116pt}}
\multiput(374.00,250.17)(10.132,6.000){2}{\rule{0.450pt}{0.400pt}}
\multiput(386.00,257.59)(1.123,0.482){9}{\rule{0.967pt}{0.116pt}}
\multiput(386.00,256.17)(10.994,6.000){2}{\rule{0.483pt}{0.400pt}}
\multiput(399.00,263.59)(1.033,0.482){9}{\rule{0.900pt}{0.116pt}}
\multiput(399.00,262.17)(10.132,6.000){2}{\rule{0.450pt}{0.400pt}}
\multiput(411.00,269.59)(1.033,0.482){9}{\rule{0.900pt}{0.116pt}}
\multiput(411.00,268.17)(10.132,6.000){2}{\rule{0.450pt}{0.400pt}}
\multiput(423.00,275.59)(1.123,0.482){9}{\rule{0.967pt}{0.116pt}}
\multiput(423.00,274.17)(10.994,6.000){2}{\rule{0.483pt}{0.400pt}}
\multiput(436.00,281.59)(1.033,0.482){9}{\rule{0.900pt}{0.116pt}}
\multiput(436.00,280.17)(10.132,6.000){2}{\rule{0.450pt}{0.400pt}}
\multiput(448.00,287.59)(1.033,0.482){9}{\rule{0.900pt}{0.116pt}}
\multiput(448.00,286.17)(10.132,6.000){2}{\rule{0.450pt}{0.400pt}}
\multiput(460.00,293.59)(0.950,0.485){11}{\rule{0.843pt}{0.117pt}}
\multiput(460.00,292.17)(11.251,7.000){2}{\rule{0.421pt}{0.400pt}}
\multiput(473.00,300.59)(1.033,0.482){9}{\rule{0.900pt}{0.116pt}}
\multiput(473.00,299.17)(10.132,6.000){2}{\rule{0.450pt}{0.400pt}}
\multiput(485.00,306.59)(1.123,0.482){9}{\rule{0.967pt}{0.116pt}}
\multiput(485.00,305.17)(10.994,6.000){2}{\rule{0.483pt}{0.400pt}}
\multiput(498.00,312.59)(1.033,0.482){9}{\rule{0.900pt}{0.116pt}}
\multiput(498.00,311.17)(10.132,6.000){2}{\rule{0.450pt}{0.400pt}}
\multiput(510.00,318.59)(1.033,0.482){9}{\rule{0.900pt}{0.116pt}}
\multiput(510.00,317.17)(10.132,6.000){2}{\rule{0.450pt}{0.400pt}}
\multiput(522.00,324.59)(1.123,0.482){9}{\rule{0.967pt}{0.116pt}}
\multiput(522.00,323.17)(10.994,6.000){2}{\rule{0.483pt}{0.400pt}}
\multiput(535.00,330.59)(1.033,0.482){9}{\rule{0.900pt}{0.116pt}}
\multiput(535.00,329.17)(10.132,6.000){2}{\rule{0.450pt}{0.400pt}}
\multiput(547.00,336.59)(1.123,0.482){9}{\rule{0.967pt}{0.116pt}}
\multiput(547.00,335.17)(10.994,6.000){2}{\rule{0.483pt}{0.400pt}}
\multiput(560.00,342.59)(1.033,0.482){9}{\rule{0.900pt}{0.116pt}}
\multiput(560.00,341.17)(10.132,6.000){2}{\rule{0.450pt}{0.400pt}}
\multiput(572.00,348.59)(0.874,0.485){11}{\rule{0.786pt}{0.117pt}}
\multiput(572.00,347.17)(10.369,7.000){2}{\rule{0.393pt}{0.400pt}}
\multiput(584.00,355.59)(1.123,0.482){9}{\rule{0.967pt}{0.116pt}}
\multiput(584.00,354.17)(10.994,6.000){2}{\rule{0.483pt}{0.400pt}}
\multiput(597.00,361.59)(1.033,0.482){9}{\rule{0.900pt}{0.116pt}}
\multiput(597.00,360.17)(10.132,6.000){2}{\rule{0.450pt}{0.400pt}}
\multiput(609.00,367.59)(1.123,0.482){9}{\rule{0.967pt}{0.116pt}}
\multiput(609.00,366.17)(10.994,6.000){2}{\rule{0.483pt}{0.400pt}}
\multiput(622.00,373.59)(1.033,0.482){9}{\rule{0.900pt}{0.116pt}}
\multiput(622.00,372.17)(10.132,6.000){2}{\rule{0.450pt}{0.400pt}}
\multiput(634.00,379.59)(1.033,0.482){9}{\rule{0.900pt}{0.116pt}}
\multiput(634.00,378.17)(10.132,6.000){2}{\rule{0.450pt}{0.400pt}}
\multiput(646.00,385.59)(1.123,0.482){9}{\rule{0.967pt}{0.116pt}}
\multiput(646.00,384.17)(10.994,6.000){2}{\rule{0.483pt}{0.400pt}}
\multiput(659.00,391.59)(1.033,0.482){9}{\rule{0.900pt}{0.116pt}}
\multiput(659.00,390.17)(10.132,6.000){2}{\rule{0.450pt}{0.400pt}}
\multiput(671.00,397.59)(1.033,0.482){9}{\rule{0.900pt}{0.116pt}}
\multiput(671.00,396.17)(10.132,6.000){2}{\rule{0.450pt}{0.400pt}}
\multiput(683.00,403.59)(0.950,0.485){11}{\rule{0.843pt}{0.117pt}}
\multiput(683.00,402.17)(11.251,7.000){2}{\rule{0.421pt}{0.400pt}}
\multiput(696.00,410.59)(1.033,0.482){9}{\rule{0.900pt}{0.116pt}}
\multiput(696.00,409.17)(10.132,6.000){2}{\rule{0.450pt}{0.400pt}}
\multiput(708.00,416.59)(1.123,0.482){9}{\rule{0.967pt}{0.116pt}}
\multiput(708.00,415.17)(10.994,6.000){2}{\rule{0.483pt}{0.400pt}}
\multiput(721.00,422.59)(1.033,0.482){9}{\rule{0.900pt}{0.116pt}}
\multiput(721.00,421.17)(10.132,6.000){2}{\rule{0.450pt}{0.400pt}}
\multiput(733.00,428.59)(1.033,0.482){9}{\rule{0.900pt}{0.116pt}}
\multiput(733.00,427.17)(10.132,6.000){2}{\rule{0.450pt}{0.400pt}}
\multiput(745.00,434.59)(1.123,0.482){9}{\rule{0.967pt}{0.116pt}}
\multiput(745.00,433.17)(10.994,6.000){2}{\rule{0.483pt}{0.400pt}}
\multiput(758.00,440.59)(1.033,0.482){9}{\rule{0.900pt}{0.116pt}}
\multiput(758.00,439.17)(10.132,6.000){2}{\rule{0.450pt}{0.400pt}}
\multiput(770.00,446.59)(1.123,0.482){9}{\rule{0.967pt}{0.116pt}}
\multiput(770.00,445.17)(10.994,6.000){2}{\rule{0.483pt}{0.400pt}}
\multiput(783.00,452.59)(1.033,0.482){9}{\rule{0.900pt}{0.116pt}}
\multiput(783.00,451.17)(10.132,6.000){2}{\rule{0.450pt}{0.400pt}}
\multiput(795.00,458.59)(1.033,0.482){9}{\rule{0.900pt}{0.116pt}}
\multiput(795.00,457.17)(10.132,6.000){2}{\rule{0.450pt}{0.400pt}}
\multiput(807.00,464.59)(0.950,0.485){11}{\rule{0.843pt}{0.117pt}}
\multiput(807.00,463.17)(11.251,7.000){2}{\rule{0.421pt}{0.400pt}}
\multiput(820.00,471.59)(1.033,0.482){9}{\rule{0.900pt}{0.116pt}}
\multiput(820.00,470.17)(10.132,6.000){2}{\rule{0.450pt}{0.400pt}}
\multiput(832.00,477.59)(1.033,0.482){9}{\rule{0.900pt}{0.116pt}}
\multiput(832.00,476.17)(10.132,6.000){2}{\rule{0.450pt}{0.400pt}}
\multiput(844.00,483.59)(1.123,0.482){9}{\rule{0.967pt}{0.116pt}}
\multiput(844.00,482.17)(10.994,6.000){2}{\rule{0.483pt}{0.400pt}}
\multiput(857.00,489.59)(1.033,0.482){9}{\rule{0.900pt}{0.116pt}}
\multiput(857.00,488.17)(10.132,6.000){2}{\rule{0.450pt}{0.400pt}}
\multiput(869.00,495.59)(1.123,0.482){9}{\rule{0.967pt}{0.116pt}}
\multiput(869.00,494.17)(10.994,6.000){2}{\rule{0.483pt}{0.400pt}}
\multiput(882.00,501.59)(1.033,0.482){9}{\rule{0.900pt}{0.116pt}}
\multiput(882.00,500.17)(10.132,6.000){2}{\rule{0.450pt}{0.400pt}}
\multiput(894.00,507.59)(1.033,0.482){9}{\rule{0.900pt}{0.116pt}}
\multiput(894.00,506.17)(10.132,6.000){2}{\rule{0.450pt}{0.400pt}}
\multiput(906.00,513.59)(1.123,0.482){9}{\rule{0.967pt}{0.116pt}}
\multiput(906.00,512.17)(10.994,6.000){2}{\rule{0.483pt}{0.400pt}}
\multiput(919.00,519.59)(0.874,0.485){11}{\rule{0.786pt}{0.117pt}}
\multiput(919.00,518.17)(10.369,7.000){2}{\rule{0.393pt}{0.400pt}}
\multiput(931.00,526.59)(1.123,0.482){9}{\rule{0.967pt}{0.116pt}}
\multiput(931.00,525.17)(10.994,6.000){2}{\rule{0.483pt}{0.400pt}}
\multiput(944.00,532.59)(1.033,0.482){9}{\rule{0.900pt}{0.116pt}}
\multiput(944.00,531.17)(10.132,6.000){2}{\rule{0.450pt}{0.400pt}}
\multiput(956.00,538.59)(1.033,0.482){9}{\rule{0.900pt}{0.116pt}}
\multiput(956.00,537.17)(10.132,6.000){2}{\rule{0.450pt}{0.400pt}}
\multiput(968.00,544.59)(1.123,0.482){9}{\rule{0.967pt}{0.116pt}}
\multiput(968.00,543.17)(10.994,6.000){2}{\rule{0.483pt}{0.400pt}}
\multiput(981.00,550.59)(1.033,0.482){9}{\rule{0.900pt}{0.116pt}}
\multiput(981.00,549.17)(10.132,6.000){2}{\rule{0.450pt}{0.400pt}}
\multiput(993.00,556.59)(1.033,0.482){9}{\rule{0.900pt}{0.116pt}}
\multiput(993.00,555.17)(10.132,6.000){2}{\rule{0.450pt}{0.400pt}}
\multiput(1005.00,562.59)(1.123,0.482){9}{\rule{0.967pt}{0.116pt}}
\multiput(1005.00,561.17)(10.994,6.000){2}{\rule{0.483pt}{0.400pt}}
\multiput(1018.00,568.59)(1.033,0.482){9}{\rule{0.900pt}{0.116pt}}
\multiput(1018.00,567.17)(10.132,6.000){2}{\rule{0.450pt}{0.400pt}}
\multiput(1030.00,574.59)(0.950,0.485){11}{\rule{0.843pt}{0.117pt}}
\multiput(1030.00,573.17)(11.251,7.000){2}{\rule{0.421pt}{0.400pt}}
\multiput(1043.00,581.59)(1.033,0.482){9}{\rule{0.900pt}{0.116pt}}
\multiput(1043.00,580.17)(10.132,6.000){2}{\rule{0.450pt}{0.400pt}}
\multiput(1055.00,587.59)(1.033,0.482){9}{\rule{0.900pt}{0.116pt}}
\multiput(1055.00,586.17)(10.132,6.000){2}{\rule{0.450pt}{0.400pt}}
\multiput(1067.00,593.59)(1.123,0.482){9}{\rule{0.967pt}{0.116pt}}
\multiput(1067.00,592.17)(10.994,6.000){2}{\rule{0.483pt}{0.400pt}}
\multiput(1080.00,599.59)(1.033,0.482){9}{\rule{0.900pt}{0.116pt}}
\multiput(1080.00,598.17)(10.132,6.000){2}{\rule{0.450pt}{0.400pt}}
\multiput(1092.00,605.59)(1.123,0.482){9}{\rule{0.967pt}{0.116pt}}
\multiput(1092.00,604.17)(10.994,6.000){2}{\rule{0.483pt}{0.400pt}}
\multiput(1105.00,611.59)(1.033,0.482){9}{\rule{0.900pt}{0.116pt}}
\multiput(1105.00,610.17)(10.132,6.000){2}{\rule{0.450pt}{0.400pt}}
\multiput(1117.00,617.59)(1.033,0.482){9}{\rule{0.900pt}{0.116pt}}
\multiput(1117.00,616.17)(10.132,6.000){2}{\rule{0.450pt}{0.400pt}}
\multiput(1129.00,623.59)(1.123,0.482){9}{\rule{0.967pt}{0.116pt}}
\multiput(1129.00,622.17)(10.994,6.000){2}{\rule{0.483pt}{0.400pt}}
\multiput(1142.00,629.59)(0.874,0.485){11}{\rule{0.786pt}{0.117pt}}
\multiput(1142.00,628.17)(10.369,7.000){2}{\rule{0.393pt}{0.400pt}}
\multiput(1154.00,636.59)(1.123,0.482){9}{\rule{0.967pt}{0.116pt}}
\multiput(1154.00,635.17)(10.994,6.000){2}{\rule{0.483pt}{0.400pt}}
\multiput(1167.00,642.59)(1.033,0.482){9}{\rule{0.900pt}{0.116pt}}
\multiput(1167.00,641.17)(10.132,6.000){2}{\rule{0.450pt}{0.400pt}}
\multiput(1179.00,648.59)(1.033,0.482){9}{\rule{0.900pt}{0.116pt}}
\multiput(1179.00,647.17)(10.132,6.000){2}{\rule{0.450pt}{0.400pt}}
\multiput(1191.00,654.59)(1.123,0.482){9}{\rule{0.967pt}{0.116pt}}
\multiput(1191.00,653.17)(10.994,6.000){2}{\rule{0.483pt}{0.400pt}}
\multiput(1204.00,660.59)(1.033,0.482){9}{\rule{0.900pt}{0.116pt}}
\multiput(1204.00,659.17)(10.132,6.000){2}{\rule{0.450pt}{0.400pt}}
\multiput(1216.00,666.59)(1.033,0.482){9}{\rule{0.900pt}{0.116pt}}
\multiput(1216.00,665.17)(10.132,6.000){2}{\rule{0.450pt}{0.400pt}}
\multiput(1228.00,672.59)(1.123,0.482){9}{\rule{0.967pt}{0.116pt}}
\multiput(1228.00,671.17)(10.994,6.000){2}{\rule{0.483pt}{0.400pt}}
\multiput(1241.00,678.59)(1.033,0.482){9}{\rule{0.900pt}{0.116pt}}
\multiput(1241.00,677.17)(10.132,6.000){2}{\rule{0.450pt}{0.400pt}}
\multiput(1253.00,684.59)(1.123,0.482){9}{\rule{0.967pt}{0.116pt}}
\multiput(1253.00,683.17)(10.994,6.000){2}{\rule{0.483pt}{0.400pt}}
\multiput(1266.00,690.59)(0.874,0.485){11}{\rule{0.786pt}{0.117pt}}
\multiput(1266.00,689.17)(10.369,7.000){2}{\rule{0.393pt}{0.400pt}}
\multiput(1278.00,697.59)(1.033,0.482){9}{\rule{0.900pt}{0.116pt}}
\multiput(1278.00,696.17)(10.132,6.000){2}{\rule{0.450pt}{0.400pt}}
\multiput(1290.00,703.59)(1.123,0.482){9}{\rule{0.967pt}{0.116pt}}
\multiput(1290.00,702.17)(10.994,6.000){2}{\rule{0.483pt}{0.400pt}}
\multiput(1303.00,709.59)(1.033,0.482){9}{\rule{0.900pt}{0.116pt}}
\multiput(1303.00,708.17)(10.132,6.000){2}{\rule{0.450pt}{0.400pt}}
\multiput(1315.00,715.59)(1.123,0.482){9}{\rule{0.967pt}{0.116pt}}
\multiput(1315.00,714.17)(10.994,6.000){2}{\rule{0.483pt}{0.400pt}}
\multiput(1328.00,721.59)(1.033,0.482){9}{\rule{0.900pt}{0.116pt}}
\multiput(1328.00,720.17)(10.132,6.000){2}{\rule{0.450pt}{0.400pt}}
\multiput(1340.00,727.59)(1.033,0.482){9}{\rule{0.900pt}{0.116pt}}
\multiput(1340.00,726.17)(10.132,6.000){2}{\rule{0.450pt}{0.400pt}}
\multiput(1352.00,733.59)(1.123,0.482){9}{\rule{0.967pt}{0.116pt}}
\multiput(1352.00,732.17)(10.994,6.000){2}{\rule{0.483pt}{0.400pt}}
\multiput(1365.00,739.59)(1.033,0.482){9}{\rule{0.900pt}{0.116pt}}
\multiput(1365.00,738.17)(10.132,6.000){2}{\rule{0.450pt}{0.400pt}}
\multiput(1377.00,745.59)(0.874,0.485){11}{\rule{0.786pt}{0.117pt}}
\multiput(1377.00,744.17)(10.369,7.000){2}{\rule{0.393pt}{0.400pt}}
\multiput(1389.00,752.59)(1.123,0.482){9}{\rule{0.967pt}{0.116pt}}
\multiput(1389.00,751.17)(10.994,6.000){2}{\rule{0.483pt}{0.400pt}}
\multiput(1402.00,758.59)(1.033,0.482){9}{\rule{0.900pt}{0.116pt}}
\multiput(1402.00,757.17)(10.132,6.000){2}{\rule{0.450pt}{0.400pt}}
\multiput(1414.00,764.59)(1.123,0.482){9}{\rule{0.967pt}{0.116pt}}
\multiput(1414.00,763.17)(10.994,6.000){2}{\rule{0.483pt}{0.400pt}}
\multiput(1427.00,770.59)(1.033,0.482){9}{\rule{0.900pt}{0.116pt}}
\multiput(1427.00,769.17)(10.132,6.000){2}{\rule{0.450pt}{0.400pt}}
\put(131.0,131.0){\rule[-0.200pt]{0.400pt}{155.380pt}}
\put(131.0,131.0){\rule[-0.200pt]{315.097pt}{0.400pt}}
\put(1439.0,131.0){\rule[-0.200pt]{0.400pt}{155.380pt}}
\put(131.0,776.0){\rule[-0.200pt]{315.097pt}{0.400pt}}
\end{picture}


Imediatamente nota-se que existe uma \textit{Superlinearidade} no \textit{speedup} dos programas em uma instância em particular: Quando o número de nós é igual ao número total de núcleos físicos de processamento. A hipótese deste fenômeno ter ocorrido se dá ao modo em que o processador i7 se comporta quando sobre uma grande carga de processos ou com o modo em que o sistema operacional escalona os processos quando nesta instância em específica. Testes de outras instâncias não registrados mostram um padrão neste fenômeno em que o speedup superlinear é diretamente relacionado a este número de processos(4 por máquina).

Nota-se também o fenômeno mais esperado de que, ao passar do número de processadores físicos, ocorre uma queda na eficiência de processamento pois o sistema estará passando uma grande quantidade de tempo escalonando os processos - ocorrendo um slowdown perceptível quando o número de processos começa a ficar muito maior do que o número de núcleos de processamento.

Atribui-se o alto desempenho ao fato de que as mensagens passadas são pequenas e rapidamente processadas entre nós; Além disso, os computadores se encontravam fisicamente próximos e é esperado que uma rede mais congestionada ou esparsa, ou mesmo um aumento no número de máquinas, causaria uma perda de eficácia em relação às instâncias testadas. Mesmo assim, o ganho de tempo no processamento distribuído para uma instância suficientemente grande sobrepõe qualquer perda de eficácia gerada por uma rede esparsa.


A implementação do projeto está disponível em \url{https://github.com/Barbiero/KnapsackAntSystem/tree/mpibranch}.